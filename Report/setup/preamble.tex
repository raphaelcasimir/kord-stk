%  A simple AAU report template.
%  2014-09-13 v. 1.1.0
%  Copyright 2010-2014 by Jesper Kjær Nielsen <jkn@es.aau.dk>
%
%  This is free software: you can redistribute it and/or modify
%  it under the terms of the GNU General Public License as published by
%  the Free Software Foundation, either version 3 of the License, or
%  (at your option) any later version.
%
%  This is distributed in the hope that it will be useful,
%  but WITHOUT ANY WARRANTY; without even the implied warranty of
%  MERCHANTABILITY or FITNESS FOR A PARTICULAR PURPOSE.  See the
%  GNU General Public License for more details.
%
%  You can find the GNU General Public License at <http://www.gnu.org/licenses/>.
%
\documentclass[11pt,twoside,a4paper,openright]{report}

%%%%%%%%%%%%%%%%%%%%%%%%%%%%%%%%%%%%%%%%%%%%%%%%
% Language, Encoding and Fonts
% http://en.wikibooks.org/wiki/LaTeX/Internationalization
%%%%%%%%%%%%%%%%%%%%%%%%%%%%%%%%%%%%%%%%%%%%%%%%
% Select encoding of your inputs. Depends on
% your operating system and its default input
% encoding. Typically, you should use
%   Linux  : utf8 (most modern Linux distributions)
%            latin1 
%   Windows: ansinew
%            latin1 (works in most cases)
%   Mac    : applemac
% Notice that you can manually change the input
% encoding of your files by selecting "save as"
% an select the desired input encoding. 
\usepackage[utf8]{inputenc}
% Make latex understand and use the typographic
% rules of the language used in the document.
\usepackage[english,danish]{babel}
% Use the vector font Latin Modern which is going
% to be the default font in latex in the future.
\usepackage{lmodern}
% Choose the font encoding
\usepackage[T1]{fontenc}
% For checkmarks: \cmark and crossmarks: \xmark
\usepackage{pifont}
	\newcommand{\cmark}{\ding{51}}%
	\newcommand{\xmark}{\ding{55}}%
%%%%%%%%%%%%%%%%%%%%%%%%%%%%%%%%%%%%%%%%%%%%%%%%
% Graphics and Tables
% http://en.wikibooks.org/wiki/LaTeX/Importing_Graphics
% http://en.wikibooks.org/wiki/LaTeX/Tables
% http://en.wikibooks.org/wiki/LaTeX/Colors
%%%%%%%%%%%%%%%%%%%%%%%%%%%%%%%%%%%%%%%%%%%%%%%%
% load a colour package
\usepackage[table,dvipsnames]{xcolor}
\definecolor{aaublue}{RGB}{33,26,82}% dark blue
\definecolor{lightGrey}{RGB}{240,240,240}% 
% The standard graphics inclusion package
\usepackage{graphicx}
% Load package to convert eps-files to use as figures
\usepackage{epstopdf}
% Set up how figure and table captions are displayed
\usepackage{caption}
\captionsetup{%
  font=footnotesize,% set font size to footnotesize
  labelfont=bf % bold label (e.g., Figure 3.2) font
}
\usepackage{wrapfig}
% For subfigures
\usepackage{subcaption}
% Make the standard latex tables look so much better
\usepackage{array,booktabs}
% Enable the use of frames around, e.g., theorems
% The framed package is used in the example environment
\usepackage{framed}
\bibliographystyle{unsrt}
% Afstand mellem listepunkter og tilføjelse af resume funktion til lister: \begin{enumerate}[resume]
\usepackage{enumitem}
\setlist{itemsep=-2pt}


% Tilføjer mulighed for at lave enkelte sider i landskab.
\usepackage{lscape}

\newcounter{listcounter}
%%%%%%%%%%%%%%%%%%%%%%%%%%%%%%%%%%%%%%%%%%%%%%%%
% Mathematics
% http://en.wikibooks.org/wiki/LaTeX/Mathematics
%%%%%%%%%%%%%%%%%%%%%%%%%%%%%%%%%%%%%%%%%%%%%%%%
% Defines new environments such as equation,
% align and split 
\usepackage{amsmath}
\usepackage{amsfonts}
% Adds new math symbols
\usepackage{amssymb}
% Use theorems in your document
% The ntheorem package is also used for the example environment
% When using thmmarks, amsmath must be an option as well. Otherwise \eqref doesn't work anymore.
\usepackage[framed,amsmath,thmmarks]{ntheorem}
% Laplace Transform sign
\usepackage{mathrsfs}

% Tilføjer \degree symbol
\usepackage{textcomp}
\usepackage{gensymb}

% Fjerner mellemrum efter komma i formler.
%\usepackage{icomma}

% Packages for SI units
\usepackage[binary-units]{siunitx}
% Format SI units as italic in italic texts
\sisetup{detect-all}

%Define new SI units
\DeclareSIUnit{\belm}{Bm}
\DeclareSIUnit{\belsm}{Bsm}
\DeclareSIUnit{\samplePerSecond}{sps}
\DeclareSIUnit{\beli}{Bi}
\DeclareSIUnit{\step}{step}
\DeclareSIUnit{\msps}{MSPS}

% Argument til amsmath der gør parenteser uden om parenteser pænere ved brug af \right og \left kommandoerne
\delimitershortfall=-1pt

%%%%%%%%%%%%%%%%%%%%%%%%%%%%%%%%%%%%%%%%%%%%%%%%
% Page Layout
% http://en.wikibooks.org/wiki/LaTeX/Page_Layout
%%%%%%%%%%%%%%%%%%%%%%%%%%%%%%%%%%%%%%%%%%%%%%%%
% Change margins, papersize, etc of the document
\usepackage[
  inner=28mm,% left margin on an odd page
  outer=41mm,% right margin on an odd page
  ]{geometry}
% Modify how \chapter, \section, etc. look
% The titlesec package is very configureable
\usepackage[explicit]{titlesec}
%\titleformat*{\section}{\normalfont\Large\bfseries\color{aaublue}}
%\titleformat*{\subsection}{\normalfont\large\bfseries\color{aaublue}}
%\titleformat*{\subsubsection}{\normalfont\normalsize\bfseries\color{aaublue}}
%\titleformat*{\paragraph}{\normalfont\normalsize\bfseries\color{aaublue}}
%\titleformat*{\subparagraph}{\normalfont\normalsize\bfseries\color{aaublue}}
\usepackage{calc}

% Spacing omkring kapiteloverskrift
\titlespacing*{\chapter}{0pt}{40pt}{50pt}

% Overskrift med stort nummer til venstre og titel til højre
%\newlength\chapnumb
%\setlength{\chapnumb}{1.5cm}
%\titleformat{\chapter}[block]
%{\normalfont\bfseries}{}{0pt}
%{\parbox[b]{\chapnumb}{%
	  %\fontsize{2cm}{0}\selectfont\thechapter}%
  %\parbox[b]{\dimexpr\textwidth-\chapnumb\relax}{%
    %\raggedleft%
    %\hfill{\Huge#1}\\
    %\rule{\dimexpr\textwidth-\chapnumb\relax}{.5pt}}}
%\titleformat{name=\chapter,numberless}[block]
%{\normalfont\bfseries}{}{0pt}
	%{\Huge#1}

% Clear empty pages between chapters
\let\origdoublepage\cleardoublepage
\newcommand{\clearemptydoublepage}{%
  \clearpage
  {\pagestyle{empty}\origdoublepage}%
}
\let\cleardoublepage\clearemptydoublepage

% Change the headers and footers
\usepackage{fancyhdr}
\pagestyle{fancy}
\fancyhf{} %delete everything
\renewcommand{\headrulewidth}{0pt} %remove the horizontal line in the header
\fancyhead[RE]{\color{black}\small\nouppercase\leftmark} %even page - chapter title
\fancyhead[LO]{\color{black}\small\nouppercase\rightmark} %uneven page - section title
\fancyhead[LE,RO]{\thepage} %page number on all pages
% Do not stretch the content of a page. Instead,
% insert white space at the bottom of the page
\raggedbottom
% Enable arithmetics with length. Useful when
% typesetting the layout.

\setlength{\headheight}{14pt}

% Raise penalties for bastards
\widowpenalty=10000
\clubpenalty=10000

%%%%%%%%%%%%%%%%%%%%%%%%%%%%%%%%%%%%%%%%%%%%%%%%
% Table of Contents
% http://en.wikibooks.org/wiki/LaTeX/Bibliography_Management
%%%%%%%%%%%%%%%%%%%%%%%%%%%%%%%%%%%%%%%%%%%%%%%%
% Add additional commands for Table of Contents
\usepackage{bookmark}

{\setcounter{tocdepth}{1}}

% Control of space between items in Table of Contents
\usepackage[titles]{tocloft}
\setlength{\cftbeforepartskip}{10pt}
\setlength{\cftbeforechapskip}{4pt}
\setlength{\cftbeforesecskip}{2pt}
%%%%%%%%%%%%%%%%%%%%%%%%%%%%%%%%%%%%%%%%%%%%%%%%
% Bibliography
% http://en.wikibooks.org/wiki/LaTeX/Bibliography_Management
%%%%%%%%%%%%%%%%%%%%%%%%%%%%%%%%%%%%%%%%%%%%%%%%
% Add the \citep{key} command which display a
% reference as [author, year]
\usepackage[square,numbers]{natbib}
%%%%%%%%%%%%%%%%%%%%%%%%%%%%%%%%%%%%%%%%%%%%%%%%
% Misc
%%%%%%%%%%%%%%%%%%%%%%%%%%%%%%%%%%%%%%%%%%%%%%%%
% Add bibliography and index to the table of
% contents
\usepackage[nottoc]{tocbibind}
% Add the command \pageref{LastPage} which refers to the
% page number of the last page
\usepackage{lastpage}
\usepackage[
%  disable, %turn off todonotes
  colorinlistoftodos, %enable a coloured square in the list of todos
  textwidth=\marginparwidth, %set the width of the todonotes
  textsize=scriptsize, %size of the text in the todonotes
  ]{todonotes}

% Add command \includepdf to add a whole pdf page to document
\usepackage{pdfpages}

% String manipulation
\usepackage{xstring,xifthen}

% Tikz package for drawing nice figures
\usepackage{tikz}

% Code syntax highlight
%\usepackage{listings}
%
%%%
%\lstset{breaklines=true,
%		breakatwhitespace=true,
%		commentstyle=\color{ForestGreen},
%		numbers=left,
%		frame=tb,
%		numberstyle=\tiny\color{black},
%    	keywordstyle=\color{blue},%
%		basicstyle=\footnotesize\ttfamily,
%        showstringspaces=false,
%		}
%\renewcommand{\lstlistingname}{Code Snippet}
%
%

\usepackage{listings}

\definecolor{dkgreen}{rgb}{0,0.6,0}
\definecolor{gray}{rgb}{0.5,0.5,0.5}
\definecolor{mauve}{rgb}{0.58,0,0.82}

\lstset{
  frame=tbs,
  language=C,
  aboveskip=3mm,
  belowskip=3mm,
  showstringspaces=false,
  columns=flexible,
  basicstyle={\small\ttfamily},
  numbers=left,
  numberstyle=\tiny\color{gray},
  keywordstyle=\color{blue},
  commentstyle=\color{dkgreen},
  stringstyle=\color{mauve},
  breaklines=true,
  breakatwhitespace=true,
  tabsize=3,
  basicstyle=\footnotesize\ttfamily
}
\renewcommand{\lstlistingname}{Code Snippet}



%% Code syntax for matlab


%%%%%%%%%%%%%%%%%%%%%%%%%%%%%%%%%%%%%%%%%%%%%%%%
% Table environments
% http://en.wikibooks.org/wiki/LaTeX/Tables
%%%%%%%%%%%%%%%%%%%%%%%%%%%%%%%%%%%%%%%%%%%%%%%%
% Better table environments for stuff like table width specifier
\usepackage{tabularx}
\usepackage{multirow}
\usepackage{longtable}
%%%%%%%%%%%%%%%%%%%%%%%%%%%%%%%%%%%%%%%%%%%%%%%%
% Project info and abstract
% chapters\abstract.tex, chapters\projectinfo.tex
%%%%%%%%%%%%%%%%%%%%%%%%%%%%%%%%%%%%%%%%%%%%%%%%
% Loads project info and abstract for use in
% hypersetup
\newcommand{\projectFaculty}{%
\iflanguage{english}{%
Electronic Engineering and IT%
}{%
Elektronik og IT%
}}

\newcommand{\projectGroup}{%
Group CE6-633%
}

\newcommand{\projectSemester}{%
P6%
}

\newcommand{\projectType}{%
\iflanguage{english}{%
Project Report%
}{%
Projektrapport%
}}

\newcommand{\projectTitle}{%
%
Rocket Navigation System
%
}

\newcommand{\projectSubtitle}{%
\iflanguage{english}{%
- Subtitle -%
}{%
- Undertitel -%
}}

\newcommand{\projectTheme}{%
Control Engineering%
}

\newcommand{\projectPeriod}{%
\iflanguage{english}{%
Spring Semester 2017%
}{%
Efterårssemester 2016%
}}



\newcommand{\projectParticipants}{%
Geoffroy Sion\\
Mathias Nielsen\\
Jacob Lassen\\
Raphaël Casimir\\
Maxime Remy\\
Romain Dieleman
}

\newcommand{\projectSupervisors}{%
Kirsten Nielsen \\ Tom Pedersen
}

\newcommand{\projectCopies}{??}

\newcommand{\projectCompletion}{
?? June 2017%
}




\newcommand{\projectAbstract}{
This reports aims to investigate the inverted pendulum and a rocket launche, as balancing an inverted pendulum and following a flight path with a rocket are both unstable systems that are controlled by applying a torque on the bottom part of a long and narrow cylinder. 

First, a brief overview is given as to why the two systems have similar stability problems before a mathematical model is derived. Both models are nonlinear and are linearized and reduced to two simple models. It is found that the simple mathematical models of each systems are not identical. A single controller controlling both systems therefore cannot be made. A controller for each system is thus designed and implemented on the respective setups. 

The controller designed for the inverted pendulum was implemented on an Arduino and uses two potentiometers and a tachometer as sensors. It is found that the controller balances the pendulum satisfactorily to the specifications made. 

The controller designed for the rocket showed that it could follow the trajectory in a linear simulation based on the model. The rocket setup was built in its entirety by the group. The rocket controller was implemented but not tested due to time constraints.

In conclusion, while a rocket during flight and an inverted pendulum share similarities with instability, the models are not identical and a controller cannot be made to work with both systems. A controller showing satisfactory balancing of the inverted pendulum  was made, but the rocket controller ultimately was not tested in flight.
}

\newcommand{\projectSynopsis}{
Synopsis}




%%%%%%%%%%%%%%%%%%%%%%%%%%%%%%%%%%%%%%%%%%%%%%%%
% Hyperlinks
% http://en.wikibooks.org/wiki/LaTeX/Hyperlinks
%%%%%%%%%%%%%%%%%%%%%%%%%%%%%%%%%%%%%%%%%%%%%%%%
% Enable hyperlinks and insert info into the pdf
% file. Hypperref should be loaded as one of the 
% last packages
\usepackage{hyperref}
\hypersetup{%
	%pdfpagelabels=true,%
	plainpages=false,%
	pdfauthor={\projectGroup, \projectFaculty, \iflanguage{english}{Aalborg University}{Aalborg Universitet}},%
	pdftitle={\projectTitle},%
	pdfsubject={\projectTheme},%
	bookmarksnumbered=true,%
	colorlinks,%
	citecolor=black,%aaublue,%
	filecolor=black,%aaublue,%
	linkcolor=black,%aaublue,% you should probably change this to black before printing
	urlcolor=black,%aaublue,%
	pdfstartview=FitH,%
	bookmarksdepth=2,%
}

% Defines where URLs should break
\def\UrlBreaks{\do\/\do-\do_}
\urlstyle{same}

% Give the possibility to autoformat reference based on distance to the referenced page. Ex. \vpageref{}
\usepackage{varioref}


% Package to warn about missing references.
%\usepackage{refcheck}

%no indent 
\usepackage[parfill]{parskip}



\usepackage{tikz-timing}
\usepackage{parnotes}
