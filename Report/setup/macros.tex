%  A simple AAU report template.
%  2014-09-13 v. 1.1.0
%  Copyright 2010-2014 by Jesper Kjær Nielsen <jkn@es.aau.dk>
%
%  This is free software: you can redistribute it and/or modify
%  it under the terms of the GNU General Public License as published by
%  the Free Software Foundation, either version 3 of the License, or
%  (at your option) any later version.
%
%  This is distributed in the hope that it will be useful,
%  but WITHOUT ANY WARRANTY; without even the implied warranty of
%  MERCHANTABILITY or FITNESS FOR A PARTICULAR PURPOSE.  See the
%  GNU General Public License for more details.
%
%  You can find the GNU General Public License at <http://www.gnu.org/licenses/>.
%
%
%
% see, e.g., http://en.wikibooks.org/wiki/LaTeX/Customizing_LaTeX#New_commands
% for more information on how to create macros

%%%%%%%%%%%%%%%%%%%%%%%%%%%%%%%%%%%%%%%%%%%%%%%%
% Loads user defined variables
%%%%%%%%%%%%%%%%%%%%%%%%%%%%%%%%%%%%%%%%%%%%%%%%
\newcommand{\noSIunit}{$1$}% Definerer hvad der skal skrives hvis symbolet ikke har nogen enhed.



%%%%%%%%%%%%%%%%%%%%%%%%%%%%%%%%%%%%%%%%%%%%%%%%
% Macros for the titlepage
%%%%%%%%%%%%%%%%%%%%%%%%%%%%%%%%%%%%%%%%%%%%%%%%
%Creates the aau titlepage
\newcommand{\aautitlepage}[3]{%
  {
    %set up various length
    \ifx\titlepageleftcolumnwidth\undefined
      \newlength{\titlepageleftcolumnwidth}
      \newlength{\titlepagerightcolumnwidth}
    \fi
    \setlength{\titlepageleftcolumnwidth}{0.44\textwidth-\tabcolsep}
    \setlength{\titlepagerightcolumnwidth}{\textwidth-2\tabcolsep-\titlepageleftcolumnwidth}
    %create title page
    \thispagestyle{empty}
    \noindent%
    \begin{tabular}{@{}ll@{}}
      \parbox{\titlepageleftcolumnwidth}{
        \iflanguage{danish}{%
          \includegraphics[page=1,width=\titlepageleftcolumnwidth]{figures/aau_logo}
        }{%
          \includegraphics[page=2,width=\titlepageleftcolumnwidth]{figures/aau_logo}
        }
      } &
      \parbox{\titlepagerightcolumnwidth}{\raggedleft\sf\small
        #2
      }\bigskip\\
       #1 &
      \parbox[t]{\titlepagerightcolumnwidth}{%
        \iflanguage{danish}{%
          \textbf{Synopsis:}\smallskip\par
        }{%
          \textbf{Abstract:}\smallskip\par
        }
        \fbox{\parbox{\titlepagerightcolumnwidth-2\fboxsep-2\fboxrule}{%
          #3
        }}
      }\\
    \end{tabular}
    \vfill
    \iflanguage{danish}{%
      \noindent{\footnotesize\emph{Rapportens indhold er frit tilgængeligt, men offentliggørelse (med kildeangivelse) må kun ske efter aftale med forfatterne.}}
    }{%
      \noindent{\footnotesize\emph{The content of this report is freely available, but publication may only be pursued with reference.}}
    }
    \cleardoublepage
  }
}

%Create english project info
\newcommand{\englishprojectinfo}[8]{%
  \parbox[t]{\titlepageleftcolumnwidth}{
    \textbf{Title:}\\ #1\bigskip\par
    \textbf{Theme:}\\ #2\bigskip\par
    \textbf{Project Period:}\\ #3\bigskip\par
    \textbf{Project Group:}\\ #4\bigskip\par
    \textbf{Participants:}\\ #5\bigskip\par
    \textbf{Supervisor:}\\ #6\bigskip\par
    \textbf{Page Numbers:} \pageref{LastPage}\bigskip\par
    \textbf{Date of Completion:}\\ #8
  }
}

%Create danish project info
\newcommand{\danishprojectinfo}[8]{%
  \parbox[t]{\titlepageleftcolumnwidth}{
    \textbf{Titel:}\\ #1\bigskip\par
    \textbf{Tema:}\\ #2\bigskip\par
    \textbf{Projektperiode:}\\ #3\bigskip\par
    \textbf{Projektgruppe:}\\ #4\bigskip\par
    \textbf{Deltagere:}\\ #5\bigskip\par
    \textbf{Vejleder:}\\ #6\bigskip\par
    \textbf{Oplagstal:} #7\bigskip\par
    \textbf{Sidetal:} \pageref{LastPage}\bigskip\par
    \textbf{Afleveringsdato:}\\ #8
  }
}

%%%%%%%%%%%%%%%%%%%%%%%%%%%%%%%%%%%%%%%%%%%%%%%%
% An example environment
%%%%%%%%%%%%%%%%%%%%%%%%%%%%%%%%%%%%%%%%%%%%%%%%
\theoremheaderfont{\normalfont\bfseries}
\theorembodyfont{\normalfont}
\theoremstyle{break}
\def\theoremframecommand{{\color{aaublue!50}\vrule width 5pt \hspace{5pt}}}
\newshadedtheorem{exa}{Example}[chapter]
\newenvironment{example}[1]{%
		\begin{exa}[#1]
}{%
		\end{exa}
}


%%%%%%%%%%%%%%%%%%%%%%%%%%%%%%%%%%%%%%%%%%%%%%%%%
% Exponential function defined as upright e, \exp
%%%%%%%%%%%%%%%%%%%%%%%%%%%%%%%%%%%%%%%%%%%%%%%%%
\renewcommand{\exp}{\text{e}}


%%%%%%%%%%%%%%%%%%%%%%%%%%%%%%%%%%%%%%%%%%%%%%%%%%%
% Command \clearevenpage start chapter on even page
%%%%%%%%%%%%%%%%%%%%%%%%%%%%%%%%%%%%%%%%%%%%%%%%%%%
\makeatletter
\newcommand*{\clearevenpage}{%
  \clearpage
  \if@twoside
    \ifodd\c@page
      \hbox{}%
      \newpage
      \if@twocolumn
        \hbox{}%
        \newpage
      \fi
    \fi
  \fi
}
\makeatother

%%%%%%%%%%%%%%%%%%%%%%%%%%%%%%%%%%%%%%%%%%%%%%%%%%%%%%%
% Command \addunit. Use it to add si units to equations
%%%%%%%%%%%%%%%%%%%%%%%%%%%%%%%%%%%%%%%%%%%%%%%%%%%%%%%
\DeclareSIUnit{\mph}{mph} %Creation of miles per hour as SI unit
\DeclareSIUnit{\miles}{mi} %Creation of miles as SI unit
\DeclareSIUnit{\feet}{ft} %Creation of feet as SI unit
\DeclareSIUnit{\inch}{in} %Creation of feet as SI unit
\makeatletter
\providecommand\add@text{}
\newcommand\addunit[1]{%
  \gdef\add@text{\text{[}\ifthenelse{\equal{#1}{}}{\noSIunit}{\si{#1}}\text{]}\gdef\add@text{}}}% 
\renewcommand\tagform@[1]{%
  \maketag@@@{\llap{\add@text\quad}(\ignorespaces#1\unskip\@@italiccorr)}%
}
\makeatother

%%%%%%%%%%%%%%%%%%%%%%%%%%%%%%%%%%%%%%%%%%%%%
% Oversættelser af ord ved brug af \autoref{}
%%%%%%%%%%%%%%%%%%%%%%%%%%%%%%%%%%%%%%%%%%%%%
\addto\extrasdanish{%
  \def\figureautorefname{Figur}%
  \def\subfigureautorefname{Figur}%
  \def\tableautorefname{Tabel}%
  \def\partautorefname{Del}%
  \def\appendixautorefname{Bilag}%
  \def\equationautorefname{Ligning}%
  \def\Itemautorefname{Punkt}%
  \def\chapterautorefname{Kapitel}%
  \def\sectionautorefname{Afsnit}%
  \def\subsectionautorefname{Afsnit}%
  \def\subsubsectionautorefname{Underafsnit}%
  \def\paragraphautorefname{Delafsnit}%
  \def\Hfootnoteautorefname{Fodnote}%
  \def\AMSautorefname{Ligning}%
  \def\theoremautorefname{Sætning}%
  \def\pageautorefname{Side}%
  \def\requirementautorefname{Krav}%
}

\addto\extrasenglish{%
  \def\chapterautorefname{Chapter}%
  \def\sectionautorefname{Section}%
  \def\subsectionautorefname{Section}%
  \def\subsubsectionautorefname{Section}%
  \def\requirementautorefname{Requirement}%
}

%%%%%%%%%%%%%%%%%%%%%%%%%%%%%%%%%%%%%%%%%%%%%%%%%%%%%%%%%%%%%%%%%%%%%%
% Tilføjer kommando \fullref{} for både at referere til nummer og navn
%%%%%%%%%%%%%%%%%%%%%%%%%%%%%%%%%%%%%%%%%%%%%%%%%%%%%%%%%%%%%%%%%%%%%%
\renewcommand*{\fullref}[1]{\hyperref[{#1}]{\autoref*{#1} \nameref*{#1}}} % One single link


%%%%%%%%%%%%%%%%%%%%%%%%%%%%%%%%%%%%%%%%%%%%%%%%%%%%%%%%%%%%%%%%%%%%%%
% Tilføjer forkortelser af ord.
%%%%%%%%%%%%%%%%%%%%%%%%%%%%%%%%%%%%%%%%%%%%%%%%%%%%%%%%%%%%%%%%%%%%%%
\newcommand{\AAU}{%
\iflanguage{english}{%
Aalborg University%
}{%
Aalborg Universitet%
}}

\newcommand{\opamp}{%
\iflanguage{english}{%
operational amplifier%
}{%
operationsforstærker%
}}

\newcommand{\hifi}{%
\iflanguage{english}{%
hi-fi amplifier%
}{%
hi-fi-forstærker%
}}

%%%%%%%%%%%%%%%%%%%%%%%%%%%%%%%%%%%%%%%%%%%%%%%%%%%%%%%%%%%%%%%%%%%%%%
% Command \requirement{Add requirement here}{Add requirement argumentation here.}
%%%%%%%%%%%%%%%%%%%%%%%%%%%%%%%%%%%%%%%%%%%%%%%%%%%%%%%%%%%%%%%%%%%%%%
\def\reqPrefixName{??}% Individual prefix for the requirements
\newcounter{reqIDCounter}% Requirement counter for the subsections
\newcounter{requirement}% Absolute requirement counter
\newcounter{reqGreyCounter}% Counter for the background colour of the req.

\newcommand{\reqPrefix}[1]{% Command to define requirement prefix
\ifthenelse{\equal{#1}{}}{\reqPrefixName}{\setcounter{reqIDCounter}{0}\def\reqPrefixName{#1}}%
}

\newcommand{\reqGrey}{% Background colour
\ifnum\thereqGreyCounter=0%
\stepcounter{reqGreyCounter}%
\def\greyRow{}%
\else%
\setcounter{reqGreyCounter}{0}%
\def\greyRow{\rowcolor{lightGrey}}
\fi%
}

\makeatletter% Define requirementID for references
\newcommand{\reqLabel}{%
\protected@edef\@currentlabel{\reqPrefixName\thereqIDCounter}%
\protected@edef\@currentlabelname{Requirement}%
}\makeatother%

\newcommand{\reqHeader}{% Command to add header to requirementtabular
\setcounter{reqGreyCounter}{0}%
\par\noindent\begin{tabularx}{1\textwidth}{p{1.1cm}p{5cm}X}%
\textit{ID} & \textit{Description} & \textit{Origin}\\%
\bottomrule%
\end{tabularx}%
\par%
}%

\newcommand{\requirement}[2]{% The actual requirement macro
\refstepcounter{requirement}%
\stepcounter{reqIDCounter}%
\reqLabel%
\reqGrey%
\par\noindent\begin{tabularx}{1\textwidth}{p{1.2cm}p{5cm}X}%
\greyRow\reqPrefixName\thereqIDCounter & #1 & \ifthenelse{\equal{#2}{}}{\textit{\textbf{No argument is specified for this requirement}}}{#2}\\%
\end{tabularx}%
\par\vspace{0.2cm}%
}%

%%%%%%%%%%%%%%%%%%%%%%%%%%%%%%%%%%%%%%%%%%%%%%%%%%%%%%%%%%%%%%%%%%%%%%
% Tilføjer kommando \numnameref{labelnavn}
% Kommandoen tilføjer en reference til et afsnit med både nummer og navn.
%%%%%%%%%%%%%%%%%%%%%%%%%%%%%%%%%%%%%%%%%%%%%%%%%%%%%%%%%%%%%%%%%%%%%%
\newcommand{\numnameref}[1]{%
\hyperref[#1]{\autoref{#1}: \nameref{#1}}%
}

%%%%%%%%%%%%%%%%%%%%%%%%%%%%%%%%%%%%%%%%%%%%%%%%%%%%%%%%%%%%%%%%%%%%%%
% Defines subscript as upright text if it contains more than one character.
% 
%%%%%%%%%%%%%%%%%%%%%%%%%%%%%%%%%%%%%%%%%%%%%%%%%%%%%%%%%%%%%%%%%%%%%%

\catcode`\_=12% Makes underscore an inactive character
\begingroup\lccode`~=`\_% Loads underscore character for redefinition
\lowercase{\endgroup\def~}#1{% Start definition of underscore
\StrLen{#1}[\subscriptstringlen]% Determine length of subscript string
\ifthenelse{\subscriptstringlen=1}% Test if subscript string contain one character
{\sb{#1}}% Prints subscript as italic if only one character is present
{\sb{\mathrm{#1}}}}% Prints subscript as upright text if there are more than one character
\AtBeginDocument{\mathcode`\_=\string"8000 }% Makes underscore active only in math mode



%%%%%%%%%%%%%%%%%%%%%%%%%%%%%%%%%%%%%%%%%%%%%%%%%%%%%%%%%%%%%%%%%%%%%%
% Tilføjer kommando \startexplain, \stopexplain og \explain{}{}
% Bruges til forklaringer af ligninger.
%%%%%%%%%%%%%%%%%%%%%%%%%%%%%%%%%%%%%%%%%%%%%%%%%%%%%%%%%%%%%%%%%%%%%%
\newcounter{firstexplain}% Holder styr på om det er den første symbolforklaring

\def\startexplain{%
	\setcounter{firstexplain}{1}%
	\noindent%
	\begin{tabular}{@{}p{.06\columnwidth}p{.76\columnwidth}@{\hskip.04\columnwidth}p{.02\columnwidth}@{}}}%
\def\stopexplain{\end{tabular}\\[10pt]}%

\newcommand{\explain}[2]{%
	\ifthenelse{\thefirstexplain=1}{%
	Where:\\&#1&[\ifthenelse{\equal{#2}{}}{\noSIunit}{#2}]\\\setcounter{firstexplain}{0}
	}{%
	&#1&[\ifthenelse{\equal{#2}{}}{\noSIunit}{#2}]\\%
	}%
}%


%%%%%%%%%%%%%%%%%%%%%%%%%%%%%%%%%%%%%%%%%%%%%%%%%%%%%%%%%%%%%%%%%%%%%%
% Tilføjer kommando \hyph
% Bruges til bindestreger i ord så de stadig kan deles korrekt af Latex.
%%%%%%%%%%%%%%%%%%%%%%%%%%%%%%%%%%%%%%%%%%%%%%%%%%%%%%%%%%%%%%%%%%%%%%
\def\hyph{-\penalty0\hskip0pt\relax}


%%%%%%%%%%%%%%%%%%%%%%%%%%%%%%%%%%%%%%%%%%%%%%%%%%%%%%%%%%%%%%%%%%%%%%
% Tilføjer kommando \hex og \bin
% Bruges til at skrive hexidecimal og binære tal.
%%%%%%%%%%%%%%%%%%%%%%%%%%%%%%%%%%%%%%%%%%%%%%%%%%%%%%%%%%%%%%%%%%%%%%
\newcommand{\hex}[1]{%
	\texttt{#1$_{16}$}
}%

\newcommand{\bin}[1]{%
	\texttt{#1$_{2}$}
}%


\newcommand{\greyrow}{%
	\rowcolor{lightGrey}
}%


%% Kommando til uppercase tau
\newcommand{\Tau}{\mathcal{T}}
