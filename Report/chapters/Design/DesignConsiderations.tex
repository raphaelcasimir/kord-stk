\graphicspath{{figures/Design/IPController/}}
\chapter{Design of the Inverted Pendulum Controller}\label{sec:IPController}
The goal of the controller is to balance the stick in an upright position. 
The system is inherently unstable as is evident by the pole in the right half plane of the pole-zero plot as seen on \autoref{fig:pzip}.

\begin{figure}[htbp]
\centering
\missingfigure{}
\caption{Pole-zero plot for the inverted pendulums transfer function.}
\label{fig:pzip}
\end{figure}
\todo[inline,author=Jacob]{Missing constants from motor test to do plot.}

There's a plethora of different options for the controller to use, the simplest being a proportional controller. A simple way to check whether the proportional controller is feasible is by examining the root locus of the transfer function in figure \autoref{fig:locusIP}. The pole in the right half plane moving towards infinity is caused by the transfer function being negative and can be fixed with a negative gain as shown on figure \autoref{fig:locusNegative}.

\begin{figure}[htbp]
\centering
\missingfigure{}
\caption{Root locus of the inverted pendulums transfer function.}
\label{fig:locusIP}
\end{figure}

\begin{figure}[htbp]
\centering
\missingfigure{}
\caption{Root locus of the inverted pendulums transfer function with a negative gain.}
\label{fig:locusNegative}
\end{figure}

The proportional controller isn't feasible as the pole in the right half plane never enters the stable region even with a negative gain.

There are two options to bring the unstable pole into the stable region: The zero in 0 can be cancelled allowing the unstable pole to cross the imaginary axis along the real axis or another pole can be added in the right half plane forcing both off the real axis with a high enough gain. The second method requires a zero in the left half plane to pull the poles towards the stable region.

Cancelling zeros in 0 by adding poles can be dangerous as the steady state error will no longer be zero which is critical for a stable inverted pendulum. The system is difficult to control as the transfer function was derived with an attempt to control the angle of the stick. This is only really possible in the special case where the reference is exactly 0 at all times. Redefining the system as trying to control the position of a point on the stick in relation to the vertical axis. By controlling the position instead of the angle would mean the system would still have to balance the stick but not be forced to use a reference of 0 at all times. This could make the system easier to control.

\section{Redefining the inverted pendulums output and reference point}

The inverted pendulum model will be redefined so the output is the distance a point on the stick to the vertical axis (where the arm and stick both have angles of 0) instead of the angle of the stick. The point, $\alpha$, and the distance to the vertical axis, $x_\alpha$, are seen on \autoref{fig:modelDist}.

\begin{figure}[htbp]
\centering
\includegraphics[width=0.7\textwidth]{ModelDist}
\caption{Diagram of the distance that will be controlled instead of the angle of the stick.}
\label{fig:modelDist}
\end{figure}

The distance to the point, $\alpha$, can be described by \autoref{eq:xa}.
\begin{flalign}\label{eq:xa}
& x_\alpha(t) = l_a\sin(\theta_a(t))+l_\alpha\sin(\theta_s(t))
\end{flalign}
This is not a linear equation and needs to be linearized in order to Laplace transform it. This is done with a 1st order Taylor approximation around the equilibrium where $\theta_a=\theta_s=0$ in \autoref{eq:xaTaylor}
\begin{subequations}\label{eq:xaTaylor}
\begin{flalign}
& x_\alpha(t)\approx l_a\sin(0)+l_a\cos(0)\theta_a(t)+l_\alpha\sin(0)+l_\alpha\cos(0)\theta_s(t) \\
& x_\alpha(t)\approx l_a\theta_a(t)+l_\alpha\theta_s(t)
\end{flalign}
\end{subequations}
This will then be Laplace transformed in \autoref{eq:xaLaplace}.
\begin{flalign}\label{eq:xaLaplace}
X_\alpha(s)=l_a\Theta_a(s)+l_\alpha\Theta_s(s) 
\end{flalign}

By isolating $\Theta_s(s)$ in \autoref{eq:tfArmStick} and inserting it into \autoref{eq:xaLaplace}, the transfer function in \eqref{eq:xatf} is found. The friction part is removed per \autoref{tab:IPModelVar}.

\begin{subequations}
\begin{flalign}
& X_\alpha(s)=l_a\Theta_a(s)+l_\alpha\frac{-\frac{3l_a}{2l_s}s^2}{s^2-\frac{3g}{2l_s}}\Theta_a(s) \\
& X_\alpha(s)=\frac{l_a\left(s^2-\frac{3g}{2l_s}\right)+l_\alpha\left(-\frac{3l_a}{2l_s}s^2\right)}{s^2-\frac{3g}{2l_s}}\Theta_a(s) \\
& \frac{X_\alpha(s)}{\Theta_a(s)} = \frac{s^2\left(l_a-l_\alpha\frac{3l_a}{2l_s}\right)-l_a\frac{3g}{2l_s}}{s^2-\frac{3g}{2l_s}} \label{eq:xatf}
\end{flalign}
\end{subequations}
The transfer function still ends up with 2 zeros but it's possible to remove them by selecting the point $\alpha$ so $l_\alpha=\frac{2l_s}{3}$. Inserting this into \autoref{eq:xatf} the transfer function becomes \autoref{eq:xaTF}.
\begin{flalign}\label{eq:xaTF}
& \frac{X_\alpha(s)}{\Theta_a(s)} = \frac{-l_a\frac{3g}{2l_s}}{s^2-\frac{3g}{2l_s}}
\end{flalign}

The zeros in 0 has now been removed but the distance, $x_\alpha$, needs to be measured. This can be done by measuring the angles which was also necessary before but now use \autoref{eq:xa} to calculate the distance instead of using the angle directly. The controller for the transfer function in \autoref{eq:xaTF} can now be designed}

\section{Design of Controller for Arm to Stick Position}



\section{Controller with a 2nd right half plane pole added}
\todo[author=Jacob, inline]{A lot of this only applies for 2nd order system which this is not. I already wrote it without considering that so this section might have to be deleted but I'm leaving it for now.}
This controller adds a zero and a pole to the system and thus have three variables that needs to be chosen: The gain and the locations of the zero and the pole. The pole has to be located somewhere in the right half plane and the zero in the left. Their locations influence the root locus of the system.

To make the system as stable as possible a fast rise time and low overshoot is desirable. From the specifications the overshoot is set as maximum 10\% with no requirements for the rise time. The damping ratio, $\zeta$, can be found from the overshoot percentage, $M_p$, by \autoref{eq:IPOvershoot}.
\begin{subequations}
\begin{flalign}
& M_p=100\exp^{\frac{-\pi\zeta}{\sqrt{1-\zeta^2}}}<10\%  \\
& \zeta = \sqrt{\frac{\left(\ln{\frac{M_p}{100}}\right)^2}{\pi^2+\left(\ln{\frac{M_p}{100}}\right)^2}}  \\
& \zeta > \sqrt{\frac{\left(\ln{\frac{10\%}{100}}\right)^2}{\pi^2+\left(\ln{\frac{10\%}{100}}\right)^2}} = 0.59 \label{eq:IPOvershoot}
\end{flalign}
\end{subequations}

The damping ratio then has to be larger than 0.59 in order to have a overshoot of less than 10\% as specified. The rise time, $t_r$, is approximated by \autoref{eq:IPRisetime}.
\begin{flalign}\label{eq:IPRisetime}
& t_r =\frac{2.2}{\zeta\omega_n} 
\end{flalign}

In order to get as fast a rise time as possible the natural frequency and damping ratio both needs to be maximized. The natural frequency can be found on the root locus by the length from the poles to the origin. The damping ratio is found by the angle from the imaginary axis to the poles. As both $\zeta$ and $\omega_n$ are dependent on the pole locations it's possible that the fastest response time comes with a damping ratio below 0.59. The goal of this controller is to minimize \autoref{eq:IPRisetime} but maintaining a damping ratio above 0.59.

The location of the pole, and zero and the gain can thus be decided in order to achieve final pole locations with as far away from the origin as possible while having an angle to the imaginary axis that corresponds to $\zeta=0.59$.

%For the inverted pendulum it's also worth to consider a controller that keeps the arm at zero radians as well as it has a much harder time controlling the arm when out of the upright position. Therefore two controllers will be made: A single-input single-output (SISO) controller that only controls the angle of the stick and a single-input multiple-output (SIMO) controller that controls both the angle of the stick and the arm. 
%\todo[inline,author=Jacob]{If time allows. Otherwise just SISO.}

%\section{Single input single output controller for the inverted pendulum}
%For the SISO controller at least a zero or pole must be introduced along with the P-controller in the system in order to move the pole in the right half plane to the left half plane. For the pole in the right half plane to cross into the left half plane a pole must be added either in 0 to cancel the zero or in the right half plane forcing both off the real axis and allowing them to circumvent the zero in 0 and enter the stable region.


\chapter{Design of the Rocket and controller}
The following chapter describes the design of the rocket and its control system. The main objective is not to design the rocket, but to implement at control system that can stabilize it during launch and flight. 

\section{Rocket Design}
For the purpose of studying the problem of rocket control, a model rocket was designed and built. Since mechanical design is out of this work's scope, the engineering of this rocket will not be explained, but the CAD files will be available on the GitHub repository.\todo{Add the github link later.}
This design consists of 2 sections, that will be called "stages".


\textbf{First stage: propulsion}
\begin{itemize}[noitemsep]
	\item {A thruster / Solid Rocket Booster (SRB)}
	\item {A thrust vectoring mechanism, also known as gimbal, with two degrees of freedom}
	\item {Two servo motors for actuating the gimbal}
\end{itemize}

\textbf{Interstage:}
\begin{itemize}[noitemsep]
	\item {An empty fairing separating the propulsion stage from the electronics}
\end{itemize}
\textbf{Upper stage: control}
\begin{itemize}[noitemsep]
	\item A frame to contain the electronics.
	\item A PCB with a micro-controller.
	\item A plastic separator with anti-vibration bearings.
	\item On this separator is placed a gyroscope: the attitude sensor.
	\item A nose fairing.
\end{itemize}

\subsubsection*{Choice of thrusters for the rocket}
A thruster is a central component in all types of rocket. In the project the thruster will be chosen based on availability and lift force. The maximum weight of the rocket can not exceed 300 grams and the thruster should be able to lift this. The average thrust of the thruster should be have at least a average thrust of 3 Newton. The choice is limited to the thrusters which can be acquired within the European regulations. Through superficial research it is found that Klima 18 mm rocket motors is legal in all of Europe, and will be chosen for the thruster. The chosen thruster is the version D3-P with the specifications cf. table \ref{ThrusterValue}.

\begin{table}[]
\centering
\begin{tabular}{lll}
\hline
Parameter      & Value         & Unit \\ \hline
Total impulse  & 17,4          & [N]  \\
Average thrust & $\approx$ 3   & [N]  \\
Maximum thrust & $\approx$ 9 & [N]  \\
Burn duration  & $\approx$ 5,5 & [s]  \\
Weight         & 0,105         & [kg] \\
Length         & 0,07          & [m] 
\end{tabular}
\caption{DP-3 thruster specifications}
\label{ThrusterValue}
\end{table}
                
%http://www.modelrockets.co.uk/shop/klima-model-rocket-motors/d3-six-pack-18mm-rocket-glider-motor-p-3311.html

\subsubsection*{Physical Parameters of the Rocket}
The important factors for controlling the rocket is the physical parameters. These will effect how the rocket would transfer a input to its output. We can not control the force of the thruster, and this can will therefore not be a part of the important   		
\begin{table}[htbp]
	\centering
	\begin{tabular}{llll}
	\hline
	Piece & Parameter & Value & Unit \\ \hline
	Rocket$_{overall}$ & Height & 297 & {[}mm{]} \\
	Rocket$_{overall}$ & Weight & 0,28 & {[}kg{]} \\
	Interstage & Diameter & 67 & {[}mm{]} \\
	Thrust vectoring system & Max. angle & ? & {[}rad{]}\\
	Thrust vectoring system & Response time & ? & {[}rad/s{]}
	\end{tabular}
\caption{Parameters of the rocket.}
\label{Rocket_measurements}
\end{table}
\todo{Input the rest of the parameters.}

\startexplain
\explain{Rocket$_{overall}$ is the total weight of the system, including thruster, electronics, and rocket structure.}{}
\stopexplain

\subsection{Design of Gimbal}
s


\subsection{Choice of Sensors for the Rocket}
The following sections describes the sensors chosen for implementation in the rocket.  

As described in section \ref{sec:PRocketAnalysis} the rocket is as a system with instability problems. In the inverted pendulum these instabilities is detected trough sampling sensors and corrected trough a DC motor control system. The same parameters is considered when controlling the rocket. A form of sensor is needed to detect the orientation and position of the rocket, and a control system is needed to counteract changes from the initial trajectory.

Choosing sensors for the rocket will be weighted based on following parameters:

\begin{itemize}[noitemsep]
\item Compatibility.
\item Sampling speed.
\item Physical dimensions and weight.
\end{itemize}

Needed is a sensor for measuring:
\begin{itemize}[noitemsep]
\item Orientation.
\item Acceleration.
\item Temperature and barometric pressure.
\end{itemize}

Determining the orientation and acceleration of the rocket can be done with different types of sensor. Two commonly used is a gyroscope and a accelerometer. 


An accelerometer measures acceleration in one to three axis(x,y,z). The reference for measuring is the gravitational force. A single axis accelerometer can measure the acceleration in the direction it is orientated. And can for example be used to determine the velocity upwards flying rocket. This can as well be used to determine a travelled distance based on knowing the acceleration and time. In the case of flying a rocket a three-axis accelerometer would be the implemented, when considering that the rocket can move both lateral and vertical in its position.  


A gyroscope is on the other hand measuring the angular velocity changes in three dimensions. The difference between the accelerometer and gyroscope is that the gyroscope is capable of measuring the rate of rotation around a axis. It does not rely on a fixed reference and is commonly used in applications like drones and other flying objects. In the rocket it can be used to determine the orientation and rotation of the rocket based on measuring the rate of changes in any direction.  


Combining these gives a Inertial Measurement Unit(IMU), which is commonly used in model planes and quad-copters. The application of this is to obtain the objects position through measuring velocity, orientation, rotation with the gyroscope and accelerometer. Both types of componenta are dependent of temperature and barometric pressure, so often an IMU includes multiple types of additional sensors for calibration purposes. A choice is made to use an IMU, given that its application in similar types of flying systems verifies that it is suitable for a rocket.    	  
Some performance factors must be considered when choosing a IMU. For example is the g-force range of the IMU important. If the maximum ratings is lower than the acceleration of the rocket, then the sensor would not be able to give sufficient data at maximum acceleration. As well is the sensitivity of the accelerometer important. The rocket is a system with a high amplitude g-force when launching, and therefore a accelerometer with low sensitivity is preferable.

GY-80 is a IMU made available for use. It includes ADXL345 3-axis accelerometer, L3G4200D 3-axis gyroscope, BMP085 thermometer/barometer, and MC5883L 3-axis magnetometer. Is chosen based on its combination of components, low power consumption, precision, and operation range. Further will be described during implementation of the component.

