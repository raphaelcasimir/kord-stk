\section{Modeling of the Gear System}\label{sec:ModGearSys}
In this section, the gear system will be modeled. 

\todo [inline,author=Geoff]{Work on intro and maybe put figure}


Three transfer functions are going to be determined in this section. First, the transfer of the motor's angular velocity $\omega_m$ to the angle of the arm $\theta_a$ through the gears. Then, too transfer functions are needed for $\tau_L$: the torque of the load comes from both the torque of the arm and stick $\tau_{as}$ and, considering the frictions in the gears, from $\omega_m$ itself.


\subsection{Relation between $\Omega_m$ and $\Theta_a$}
Ratio between small wheel x, $n_x$, and big wheel x, $N_x$ is the same:
\begin{equation}
	\frac{n_x}{N_x} = \frac{n_{motor}}{N_{w_1}}
\end{equation}

\subsection{Linear Model}
\todo[inline,author=Geoff]{Why linear model?}

In order to design an effective feedback control system a linear models is a necessity. As seen previously in \autoref{} \todo[inline, author=Max]{Put the ref for this equation} $\frac{\Theta_a(s)}{\omega_m(s)}$ is already a linear relationship. So in order to get a linear system only the transfer function $\frac{\omega_m(s)}{U(s)}$ has to be linearized.

In order to do $\tau_l$ has to be rewrite in function of $U(s)$ and $\omega_m(s)$.

\subsection{Determining $\tau_l$}
As said in \autoref{sec:ModGearSys} $\tau_l(s)$ is the torque of the load. In the inverted pendulum case the load is composed of the friction and the torque necessary to counter the gravity which result in \autoref{eq:TauL}.

\begin{equation}\label{eq:TauL}
	\tau_l = \tau_f + \tau_{as}
\end{equation}

It is necessary then to derive $\tau_f$ and $\tau_{as}$ to see if they are requiring linearization.

\subsubsection*{Determining $\tau_{f}$}

\subsubsection*{Determining $\tau_{as}$}

$\tau_{as}$ is the torque necessary for the arm to counter the gravity. It can be divided into two parts. The first one $\tau_a$ is generated by the arm when it is not parallel to the gravitational force. As seen in \autoref{fig:tauA} this force can be seen as the pendulum where the mass is concentrated in the GC and has a weightless rod, so the gravitational force can be summarized as \autoref{eq:tauA}.

\begin{figure}[htbp]
	\missingfigure{A diagram showing how the gravity affects the arm}
	\caption{Diagram of the gravitational force on the arm}\label{fig:tauA}
\end{figure}

\begin{equation}\label{eq:tauA}
	\tau_a=m_a\cdot g \cdot \frac{l_a}{2} sin(\Theta_a)
\end{equation}
\startexplain
\explain{$m_a$ is the mass of the arm}{\si{\kilogram}}
\explain{$l_a$ is the length of the arm}{\si{\meter}}
\stopexplain

Since $\tau_a$ is dependent of a sin function, in the small angles, it can be linearized in \autoref{eq:tauALin}.

\begin{equation}\label{eq:tauALin}
\tau_a=m_a\cdot g \cdot \frac{l_a}{2}\cdot\Theta_a
\end{equation}