\newpage
\section{Combining the Models for the Inverted Pendulum}\label{sec:CombinedModel}
With a model for each individual part of the inverted pendulum derived, a combined model for the entire system can be made. This model will have voltage, $U_m(s)$, as input and the angle of the stick, $\Theta_s(s)$, as the output. This is done by multiplying all transfer functions together as seen in \autoref{eq:CombTF}.
\begin{flalign}
\frac{\Omega_m(s)}{U_m(s)}\cdot \frac{\Theta_a(s)}{\Omega_m(s)}\cdot \frac{\Theta_s(s)}{\Theta_a(s)}=\frac{\Theta_s(s)}{U_m(s)} \label{eq:CombTF}
\end{flalign}

By combining \autoref{eq:DCModelNoL}, \autoref{eq:thetaOmega} and \autoref{eq:tfArmStick} the model for the inverted pendulum becomes \autoref{eq:CombModel}.
\begin{flalign}
\frac{\Theta_s(s)}{U_m(s)}=\frac{K_t}{J_tR_ms+B_tR_m+K_tK_e} \cdot \frac{N^3}{s} \cdot \frac{-\frac{3l_a}{2l_s}s^2+\frac{3b_{as}}{M_sl_s^2}s}{s^2+\frac{3b_{as}}{M_sl_s^2}s-\frac{3g}{2l_s}} \label{eq:CombModel}
\end{flalign}

The values for all variables for the inverted pendulum along with their origin is shown in \autoref{tab:IPModelVar}. The friction between the arm and the stick is assumed to be zero as the joint consists of a ball bearing. 
\begin{table}[htbp]
\centering
\caption{Parameters for the inverted pendulum and their origin.}
\label{tab:IPModelVar}
\begin{tabular}{lllll}
\hline
Variable & Value & Parameter & Unit & Origin\\ \hline
$K_t$ & $29.3\cdot10^-3$ & Mechanical motor constant & \si{\newton\meter\per\ampere} & \autoref{sec:MeasKt} \\
$K_e$ & $35.5\cdot10^-3$ & Electrical motor constant & \si{\volt\per(\radian\per\second)} & \autoref{sec:MeasKe} \\
$J_m$ & $29.0\cdot 10^{-6}$ & Moment of inertia of motor & \si{\kilogram\square\meter} & \cite{datasheet:dcmotor} \\
$J_{gear}$ & $0.153\cdot 10^{-3}$ & Moment of inertia of gear & \si{\kilogram\square\meter} & \autoref{sec:MeasJg} \\
%$J_a$ & $10.3\cdot 10^{-3}$ & Moment of inertia of the arm & \si{\kilogram\square\meter} \\
%$J_s$ & $1.95\cdot 10^{-3}$ & Moment of inertia of the stick & \si{\kilogram\square\meter} \\
%$L_m$ & $1.56\cdot 10^{-4}$ & Inductance & \si{\henry} \\
$R_m$ & 0.8 & Resistance & \si{\ohm} & \autoref{sec:MeasRm} \\
$B_m$ & $0.159\cdot 10^{-3}$ & Friction in motor & \si{\newton\per(\radian\per\second)} & \autoref{sec:MeasBm} \\
$B_{gear}$ & $1.11\cdot 10^{-3}$ & Friction in the gears & \si{\newton\per(\radian\per\second)} & \cite{web:BalancingStick2008} \\
$N$   & 0.3 & Gear ratio & 1 & \autoref{GearSystemParameters} \\   
$l_a$ & 0.33 & Length of arm & \si{\meter} & \autoref{DimensionsStick} \\
$l_s$ & 0.8 & Length of stick & \si{\meter} & \autoref{DimensionsStick} \\
$M_a $ & 0.288 & Mass of arm & \si{\kilogram} & \autoref{DimensionsStick} \\
$M_s$ & 0.344 & Mass of stick & \si{\kilogram} & \autoref{DimensionsStick} \\
$b_{as}$ & 0 & Friction in arm and stick & \si{\newton\per(\radian\per\second)} & Assumption \\  
$g$ & 9.8 & Standard gravity & \si{\meter\per\square\second} & \cite{web:gravity}
\end{tabular}
\end{table}

With the mathematical model of the inverted pendulum found a controller for the system can be designed.