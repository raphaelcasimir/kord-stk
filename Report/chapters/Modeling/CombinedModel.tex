\newpage
\section{Combining the Models for the Inverted Pendulum}\label{sec:CombinedModel}
With a model for each individual part of the inverted pendulum derived, a combined model for the entire system can be made. This model will have voltage, $U_m(s)$, as input and the angle of the stick, $\Theta_s(s)$, as the output. This is done by multiplying all transfer functions together as seen in \autoref{eq:CombTF}.
\begin{flalign}
\frac{\Omega_m(s)}{U_m(s)}\cdot \frac{\Theta_a(s)}{\Omega_m(s)}\cdot \frac{\Theta_s(s)}{\Theta_a(s)}=\frac{\Theta_s(s)}{U_m(s)} \label{eq:CombTF}
\end{flalign}
In block form this is simply done by putting each model sequentially after each other as shown in \autoref{fig:CombBlock}.
\begin{figure}[htbp]
\centering
\missingfigure{Block diagram of the model for the inverted pendulum.}
\label{fig:CombBlock}
\end{figure}

By combining \autoref{eq:Omega_m}, \autoref{eq:thetaOmega} and \autoref{eq:tfArmStick} the model for the inverted pendulum becomes \autoref{eq:CombModel}.
\begin{flalign}
\frac{\Theta_s(s)}{U_m(s)}=\frac{2 K_t s}{A s^3+B s^2 + C s + D}\cdot \frac{N^3}{s} \cdot \frac{-\frac{3l_a}{2l_s}s^2+\frac{3b_{as}}{M_sl_s^2}s}{s^2+\frac{3b_{as}}{M_sl_s^2}s-\frac{3g}{2l_s}} \label{eq:CombModel}
\end{flalign}

The values for each variable for the inverted pendulum is shown in \autoref{tab:IPModelVar}. The friction between the arm and the stick is assumed to be zero as the joint consists of a ball bearing. The other variables are found in \autoref{sec:IPDesc}, \autoref{} and \autoref{}.
\begin{table}[htbp]
\centering
\caption{Parameters for the inverted pendulum.}
\label{tab:IPModelVar}
\begin{tabular}{llll}
\hline
Variable & Value & Parameter & Unit \\ \hline
$K_t$ & 0.0293 & Mechanical motor constant & \si{\newton\meter\per\ampere} \\
$K_e$ & 0.0355 & Electrical motor constant & \si{\volt\per(\radian\per\second)} \\
$J_m$ & $2.9\cdot 10^{-5}$ & Moment of inertia of the motor & \si{\kilogram\square\meter} \\
$J_{gear}$ & $0.153\cdot 10^{-3}$ & Moment of inertia of the gear & \si{\kilogram\square\meter} \\
$J_a$ & $10.3\cdot 10^{-3}$ & Moment of inertia of the arm & \si{\kilogram\square\meter} \\
$J_s$ & $1.95\cdot 10^{-3}$ & Moment of inertia of the stick & \si{\kilogram\square\meter} \\
$L_m$ & $1.42\cdot 10^{-4}$ & Inductance & \si{\henry} \\
$R_m$ & 1.02 & Resistance & \si{\ohm} \\
$B_m$ & $1.59\cdot 10^{-4}$ & Friction in motor & \si{\newton\per(\radian\per\second)} \\
$B_{gear}$ & $1.11\cdot 10^{-3}$ & Friction in motor & \si{\newton\per(\radian\per\second)} \\
$N$   & 0.027 & Gear ratio & [1] \\   
$l_a$ & 0.33 & Length of arm & \si{\meter} \\
$l_s$ & 0.4 & Length of stick & \si{\meter} \\
$M_s$ & 0.17 & Mass of stick & \si{\kilogram} \\
$b_{as}$ & 0 & Friction between arm and stick & [1] \\  
$g$ & 9.8 & Standard gravitational acceleration & \si{\meter\per\square\second} 
\end{tabular}
\end{table}
\todo[inline,author=maxime]{Ja,Js,Bgear has to be found or a reference to the previous report should be done}
Inserting the numbers from \autoref{tab:IPModelVar} the transfer function of the model for the inverted pendulum becomes \autoref{eq:CombFinalTF}.
\begin{flalign}\label{eq:CombFinalTF}
\frac{\Theta_s(s)}{U_m(s)}=
\end{flalign}

With the mathematical model of the inverted pendulum found a controller for the system can be designed.