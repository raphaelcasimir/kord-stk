\subsubsection{Determining $\frac{\Theta_a}{U_m}$ }


\todo[inline,author=Maxime]{Test to see the minimum torque to overcome dry friction}
As the motor torque combined with the gear is really large the load generated by the arm and the stick can be neglected. Indeed the motor has to use at least \SI{3.5}{\ampere} to move so it requires a torque of \SI{3.8}{\newton\meter} at the end of the gear train. On the other hand the torque load applied by the arm and stick without the momentum is around \SI{0.005}{\newton\meter} while the arm has to go at least 

Also due to their small value the inductance and the friction at joint of the arm and the stick are also neglected. Such assumptions means that the load on the motor comes then solely from the friction of the gears and the motor itself. This combined with Equations (\ref{eq:Iinserted}), (\ref{eq:TaufSimplified}) and (\ref{eq:thetaOmega}) gives \autoref{eq:begTFOaUm}.


\begin{align}\label{eq:begTFOaUm}
	s J_{m} \Omega_m(s) &=K_{t} \frac{U_m-K_e \Omega_m(s)}{R_m} \notag \\ 
	&- \Omega_m(s) (J_{gear}s+B_{gear}) - B_{m} \Omega_m(s) \addunit{1}
\end{align}

Factorizing \autoref{eq:begTFOaUm} by $\Omega_m$ and $U_m$ \autoref{eq:midTFOaUm} is obtained.

\begin{equation}\label{eq:midTFOaUm}
	\Omega_m(s)\left(s J_m+\frac{K_t K_e}{ R_m} + (J_{gear}s+B_{gear}) + B_{m}  \right)=\frac{U_m K_t}{R_m}	
\end{equation}

And so $\frac{\Omega_m}{U_m}$ is obtained in \autoref{eq:finTFOaUm}.
\begin{subequations}\label{eq:finTFOaUm}
	\begin{align}
		\Omega_m&=\Theta_a\frac{s}{N^3}\\
		\frac{\Omega_m(s)}{U_m}&=\frac{\frac{ K_t}{R_m}}{s(J_m+J_{gear})+\frac{K_t K_e +R_m (B_{gear}+B_m)}{R_m}} \\
		\frac{\Theta_a(s)}{U_m}&=\frac{N^3}{s}\frac{\frac{ K_t}{R_m}}{s(J_m+J_{gear})+\frac{K_t K_e +R_m (B_{gear}+B_m)}{R_m}}
	\end{align}
\end{subequations}