\graphicspath{{figures/design/}}
\chapter{Test Journal: Tachometer}\label{appendix:RPMTest}
\begin{table}[!h]
\begin{tabular}{l l}
\textbf{Test participants:} & Mathias  \\
\textbf{Date:}  & 10/04-2017
\end{tabular}
\end{table}

\section*{Purpose}
The purpose of the test is to determine the linearity and precision of the tachometer used in the system.
\section*{Test equipment and components}
\begin{table}[htbp]
	\centering
	\caption{List of measurement equipment and components}\label{tab_appendix:RPMSetup}
	\begin{tabularx}{\textwidth}{lXXXX}
		Name & Brand & Model & AAU-number \\ \toprule \rowcolor{lightGrey}
		Oscilloscope	& Agilent & 54621D & 33941 	\\
		Power supply	& Agilent & E3631A & 78577\\ 
		\rowcolor{lightGrey}	
		DC motor & Alsthom BBC & F9M2& 08339\\
		Tachometer & Compact \newline Instruments & A2108& 77087 \\ \rowcolor{lightGrey}
		Tachometer & Shimpo& DT-205 & 18246
	\end{tabularx}
\end{table}
\section*{Setup}
The powersupply is connected to the motor directly. Tachometer A2108 is connected to the oscilloscope to read its output voltage. Tachometer DT-215 is not connected to anything, it shows the RPM on in-built display.  
\section*{Method}
Step by step of test, maybe in enumerate
\begin{enumerate}
\item Connect motor directly to the power supply.
\item Connect A2108 tachometer to the oscilloscope.
\item Place a reflective tape on the motor axle.
\item Set holder for both tachometers so they point at the motor axle.
\item Set the power supply to 2 V.
\item Note the voltage from the oscilloscope, and times the voltage with 1000 to find the number of rotations per minute.
\item Note the Rotations Per Minute for the each tachometer.
\item Change the voltage with 1 V increments from 2 V - 10 V.
\item Note the RPM with each increment.
\end{enumerate}
\section*{Raw data}
\begin{table}[htbp]
\centering
\caption{Rotations Per Minute of the both tachometers.}
\label{RPMData}
\begin{tabular}{llll}
Voltage & DT-205 {(}RPM{)} & A2108 {(}RPM{)} & Difference {(}RPM{)} \\ \hline  \rowcolor{lightGrey}
2 V     & 204              & 202             & 2                    \\
3 V     & 506              & 502             & 4                    \\  \rowcolor{lightGrey}
4 V     & 763              & 760             & 3                    \\ 
5 V     & 1023             & 1024            & 1                    \\  \rowcolor{lightGrey}
6 V     & 1324             & 1322            & 2                    \\
7 V     & 1641             & 1640            & 1                    \\  \rowcolor{lightGrey}
8 V     & 1912             & 1913            & 1                    \\
9 V     & 2218             & 2214            & 4                    \\ \rowcolor{lightGrey}
10 V    & 2460             & 2456            & 4                    
\end{tabular}
\end{table}

\section*{Data processing}

\begin{figure}[htbp]
	\centering
	\includegraphics[width=\textwidth]{figures/appendix/Motor&GearTests/RPMTest}
	\caption{Plot of RPM found for both tachometers}\label{fig:RPMTest}
\end{figure}

\section*{Conclusion}
It its seen cf. figure \ref{fig:RPMTest} that the difference between the tachometers is so little, that the lines is on top of each other. The precision do not change with higher velocities and the A2108 tachometer is chosen, because this can output a analogue voltage which can be sampled by an Arduino.

