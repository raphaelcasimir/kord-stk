\graphicspath{{figures/design/}}
\chapter{Test Journal: Tachometer}
\begin{table}[!h]
\begin{tabular}{l l}
\textbf{Test participants:} & Mathias  \\
\textbf{Date:}  & 10/04-2017
\end{tabular}
\end{table}

\section*{Purpose}
The purpose of the test is to determine the linearity and precision of the tachometer used in the system.
\section*{Test equipment and components}
\begin{table}[htbp]
	\centering
	\caption{List of measurement equipment and components}\label{tab_appendix:RPMSetup}
	\begin{tabularx}{\textwidth}{lXXXX}
		Name & Brand & Model & AAU-number \\ \toprule \rowcolor{lightGrey}
		Oscilloscope	& Agilent & 54621D & 33941 	\\
		Powersupply	& Agilent & E3631A & 78577\\ 	
		DC motor & Alsthom BBC & F9M2& 08339\\
		Tachometer & Compact \newline Instruments & A2108& 77087 \\
		Tachometer & Shimpo& DT-205 & 18246
	\end{tabularx}
\end{table}
\section*{Setup}
The powersupply is connected to the motor directly. Tachometer A2108 is connected to the oscilloscope to read its output voltage. Tachometer DT-215 is not connected to anything, it shows the RPM on in-built display.  
\section*{Method}
Step by step of test, maybe in enumerate
\begin{enumerate}
\item one
\item two
\end{enumerate}
\section*{Raw data}
Or maybe a plot if there's many results. (Remember to save the raw data anyway.)
\section*{Data processing}
\section*{Conclusion}


