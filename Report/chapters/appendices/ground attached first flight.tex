\graphicspath{{figures/design/}}


			\chapter{Test Journal: Attached Flight}



\begin{table}[!h]
\begin{tabular}{l l}
\textbf{Test participants:} & Romain \& Raphael   \\
\textbf{Date:}  & 01/05-2017
\end{tabular}
\end{table}




	\section*{Purpose}
The purpose of the test is to experiment and approve the hardware, software and controller of the rocker in a first aptempt of a stable flight.




	\section*{Test equipment and components}
\begin{table}[h]
	\centering
	\caption{List of measurement equipment and components}\label{tab_appendix:template}

	\begin{tabularx}{\textwidth}{lXXXX}
		Name 				 									\\ \toprule \rowcolor{lightGrey}
		Rocket body		\\
		Thruster	\\ 
		Camera 	\\
		Rope and poles \\ \rowcolor{lightGrey}
	\end{tabularx}
\end{table}




	\section*{Setup}
Measurement setup is seen on \autoref{fig:DCSetup} and photo \
\begin{figure} [h]
\centering
\includegraphics[width=0.6\linewidth]{figures/frontpage.jpg}
\caption{Measurement setup.}
\label{fig:DCSetup}
\end{figure}

 The Rocket body is attached by two ground-fixed ropes on two opposite side position, around its center of pressure. The rocket has a maximum vertical movement of "unknown" meters.




	\section*{Method}
 The setup of the rocket will ennable a safe, for observers and the system, simulation of the rocket flight. The proper functionment of the controller is seen in a stable movement of the ground-fixed rocket



	\section*{Raw data}

Photo of different positions or multiple test.




	\section*{Data processing}	
The rocket becomes stable after a period of "unknown" secondes.
 
 
 

	\section*{Conclusion}

The rocket stability comes at an acceptable period of time after take-off. The requirement is then fullfilled.


