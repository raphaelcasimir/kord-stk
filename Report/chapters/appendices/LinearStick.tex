\chapter{Linearization of the Arm and Stick Model}
\label{sec:LinearStick}
This appendix will linearize \autoref{eq:unlinStick}
\begin{flalign}
J_s\ddot{\theta}_s = \frac{l_s}{2}M_s \Big( &-l_a\ddot{\theta}_a \cos(\theta_a-\theta_s)+l_a\dot{\theta}_a^2 \sin(\theta_a-\theta_s) \notag \\
&-\frac{l_s}{2}\ddot{\theta}_s -g\sin(\theta_s) \Big)-b_{as}\dot{\theta}_{as} \label{eq:unlinStick}
\end{flalign}
The linearization is made with a 1st order Taylor approximation. The linearization is done at the equilibrium point where the arm is in an upright position i.e. $\theta_s=0$. In the equilibrium point the derivatives of all the inputs and outputs are 0. In this case the inputs and outputs are $\theta_a$ and $\theta_s$ and the operating point is $\bar{\theta}_a=0$ and $\bar{\theta}_s=0$. The nonlinear model is expressed as a function of the inputs and outputs as seen in \autoref{eq:nonlinearmodel}.
\begin{flalign}\label{eq:nonlinearmodel}
f\left(\theta_a, \dot{\theta}_a, \ddot{\theta}_a, \theta_s, \ddot{\theta}_s\right)=\frac{l_s}{2}M_s \Big( &-l_a\ddot{\theta}_a \cos(\theta_a-\theta_s)+l_a\dot{\theta}_a^2 \sin(\theta_a-\theta_s) \notag \\
&-\frac{l_s}{2}\ddot{\theta}_s -g\sin(\theta_s) \Big)-b_{as}\dot{\theta}_{as}-J_s\ddot{\theta}_s
\end{flalign}

Generally all equilibriums can be found by setting \autoref{eq:nonlinearmodel} equal to 0 and the derivatives to 0 and solving for $\theta_a=\bar{\theta}_a$ and $\theta_s=\bar{\theta}_s$. For the pendulum it's easy to see the only two equilibriums are the stick pointing straight up and straight down. 

The 1st order Taylor approximation of an equation with multiple variables is seen in \autoref{eq:1stTaylor}.
\begin{flalign}
 f\left(\theta_a, \dot{\theta}_a, \ddot{\theta}_a, \theta_s, \ddot{\theta}_s\right) & \approx f\left(\bar{\theta}_a, 0, 0, \bar{\theta}_s, 0\right) + \left. \frac{\partial f}{\partial \theta_a}\right|_{(\bar{\theta}_a, \bar{\theta}_s)} \hat{\theta}_a \notag \\
& \phantom{=} + \left. \frac{\partial f}{\partial \dot{\theta}_a}\right|_{(\bar{\theta}_a, \bar{\theta}_s)} \hat{\dot{\theta}}_a + \left. \frac{\partial f}{\partial \ddot{\theta}_a}\right|_{(\bar{\theta}_a, \bar{\theta}_s)} \hat{\ddot{\theta}}_a \notag \\
& \phantom{=} + \left. \frac{\partial f}{\partial \theta_s}\right|_{(\bar{\theta}_a, \bar{\theta}_s)} \hat{\theta}_s + \left. \frac{\partial f}{\partial \ddot{\theta}_s}\right|_{(\bar{\theta}_a, \bar{\theta}_s)} \hat{\ddot{\theta}}_s \label{eq:1stTaylor}
\end{flalign}
\startexplain
	\explain{$\bar{\theta}$ denotes the angle in an operating point}{\si{\radian}}
	\explain{$\hat{\theta}$ denotes the angle of the small signal variances}{\si{\radian}}
\stopexplain

The 3 terms with sin or cos in \autoref{eq:JsShort} will be approximated individually using \autoref{eq:1stTaylor}, remembering that $\bar{\theta}=\bar{\dot{\theta}}=\bar{\ddot{\theta}}=0$ in the equilibrium.
\begin{subequations}
\begin{flalign}
 -l_a\ddot{\theta}_a\cos\left(\theta_a-\theta_s\right)  \approx & \ 0 + l_a\bar{\ddot{\theta}}_a\sin\left(\bar{\theta}_a-\bar{\theta}_s\right)\hat{\theta}_a  \notag \\ 
& -l_a\cos\left(\bar{\theta}_a-\bar{\theta}_s\right)\hat{\ddot{\theta}}_a - l_a\bar{\ddot{\theta}}_a\sin\left(\bar{\theta}_a-\bar{\theta}_s\right)\hat{\theta}_s   \\
 -l_a\ddot{\theta}_a\cos\left(\theta_a-\theta_s\right) \approx &-l_a\hat{\ddot{\theta}}_a 
\end{flalign} %\bar{\ddot{\theta}}_s
\end{subequations}
\begin{subequations}
\begin{flalign}
l_a\dot{\theta}_a^2\sin\left(\theta_a-\theta_s\right)  \approx &\ 0 + l_a\bar{\dot{\theta}}_a^2\cos\left(\bar{\theta}_a-\bar{\theta}_s\right)\hat{\theta}_a  \notag \\
& + 2l_a\bar{\dot{\theta}}_a\sin\left(\bar{\theta}_a-\bar{\theta}_s\right)\hat{\dot{\theta}}_a-l_a\bar{\dot{\theta}}_a^2\cos\left(\bar{\theta}_a-\bar{\theta}_s\right)\hat{\theta}_s   \\
 l_a\dot{\theta}_a^2\sin\left(\bar{\theta}_a-\bar{\theta}_s\right) \approx & \ 0 
\end{flalign}
\end{subequations}
\begin{subequations}
\begin{flalign}
& g\sin\left(\theta_s\right) \approx g\sin\left(\bar{\theta}_s\right) +g\cos\left(\bar{\theta}_s\right)\hat{\theta}_s \\
& g\sin\left(\theta_s\right) \approx g\hat{\theta}_s 
\end{flalign}
\end{subequations}

The linearized model then becomes \autoref{eq:LinearStick}.
\begin{flalign}
& J_s\hat{\ddot{\theta}}_s=\frac{l_s}{2}M_s\left(-l_a\hat{\ddot{\theta}}_a-\frac{l_s}{2}\hat{\ddot{\theta}}_s-g\hat{\theta}_s\right)-b_{as}\hat{\dot{\theta}}_{as} \label{eq:LinearStick}
\end{flalign}