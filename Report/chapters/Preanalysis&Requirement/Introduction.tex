\chapter{Introduction}\label{sec:Introduction}
%This project deals with the design of a stabilization system based on modelling of a rocket.
%%within the atmosphere of Earth. 
%
%On the 15th of february 2017 an Indian rocket successfully launched with a payload of 104 satellites. The rocket beat the previous payload record for a Russian rocket carrying 37 satellites. This is a accomplishment and important milestone for rockets, but in the early days rockets were not developed based on space travelling.
%
%Rockets are a technology used for hundreds of years. In the 13th century the Chinese and Mongols used fire based rockets, which had chemical mixed gunpowder as propulsion. Trough the 14th and 15th century the gunpowder improved and the rockets range increased, but was at this time considered as weapons.  
%
%The understanding of modern rockets was based on experiments with steam propulsion, as with for example the aeolipile in the 17th century. Aeolipile is a rocket-like sphere, that rotates trough directing pressurised steam through opposite pointing L-shaped nozzles\cite{web:HistoryRocket}. 
%
%This meant that the approach to rockets became more scientific. In the later 17th century Sir Isaac Newton determined his three laws of motions which led to new understanding in rocket science.
%
%Rocket was still considered as weapons at this state and the accuracy was one of the major problems. The understanding and knowledge of aerodynamics was not yet established at this point. And therefore these rockets could not be used to carry a preservable payload, due to their accuracy and stability. 
%
%
%From the 20th century and forward the use of rockets rapidly progressed, and within the last 60 years rocket have been used for travelling and exploring our atmosphere and space. The development of this has been beneficial for humans, with the launching of satellites into orbit around earth and the scientific exploration of space. Today satellites and space travelling have a wide spread use for communication, scientific research and monitoring, which benefit most people on daily basis. This was not possible without the development of stable rockets able to carry a payload\cite{web:HistoryRocket}.
%
%
%The early days problem of rockets was the short travelling distance. Therefore the focus was having better thrust to gain further range. The design of liquid fuel motors led to distance no longer being a problem. The main concern was the amount of fuel needed and the weight of the payload. The heavier the rocket is, the more fuel it needs. Adding more fuel makes the rocket heavier. Incorrect distribution of the fuel could therefore make the rocket unstable, due to the change of the rocket's center of gravity.
%
%
%Stability and inaccuracy were also some of the major concerns in 19th century rockets. The rockets were designed with the thruster placed in the top/front, similar to how fireworks are designed today. This causes instability if the thruster is not powerful enough, or the weight distribution is incorrect. As well does a disturbance as wind etc. effect the stability. But as we know from modern full scale rockets the thruster is placed at the bottom, which gives a higher risk of instability problems. When the thrust is applied from the bottom, the rocket needs to be stable in the top. If the rocket becomes unbalanced at tip, the more likely it is to deviate from its path. 
%
%
%
%\bigbreak
%Incorrect weight distribution, faulty design, and aerodynamics are factors leading to instability in rockets. These stability problems have all been faced throughout the history of rocket development, and will therefore be further examined to determine if an electronic control system can be implemented to improve the stability issues of rockets.

The stabilization of an inverted pendulum is one of the well-known example used in education to explain classical mechanics and physics. It started to appear in the 1960's with Roberge's bachelor thesis "The Mechanical Seal" \citep{sci_article:InvertedPendulumHistory}. It is now one of the main benchmark for testing nonlinear control techniques. Its principle is to balance a stick in an upright position by controlling a mechanical system.

The project aims to model and control an unstable inverted pendulum system to prove that its fundamental principle is applicable to a real life system. The flight control of a rocket during the initial stage of its launch presents similarities \citep{sci_article:InvertedPendulumHistory}. These similarities can already be intuitively seen as both aims to stabilize a stick by applying a force lower than its center of gravity. Therefore a comparison between the model and control of the inverted pendulum and the rocket will be drawn in the report. An initial problem statement will be determined to set the project research focus.  