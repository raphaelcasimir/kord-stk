\chapter{Introduction}\label{sec:Introduction}\todo{Should be revised - Mathias}
This project deals with the design of a stabilization system based on modelling of a rocket.
%within the atmosphere of Earth. 

On the 15th of february 2017 an Indian rocket successfully launched with a payload of 104 satellites. The rocket beat the previous payload record for a Russian rocket carrying 37 satellites. This is a accomplishment and important milestone for rockets, but in the early days rocket was not developed based on space travelling.

Rockets is a technology used for hundreds of years. In the 13th century the Chinese and Mongols used fire based rocket, which had chemical mixed gunpowder as propulsion. Trough the 14th and 15th century the gunpowder improved and the rockets range increased, but was at this time considered as weapons.  

The understanding of modern rockets was based on experiments with steam propulsion as with for example the aeolipile in the 17th century. Aeolipile is a rocket-like sphere, that rotates trough directing pressurised steam trough opposite pointing L-shaped nozzles\cite{web:HistoryRocket}. 

This meant that the approach to rockets became more scientific. In the later 17th century Sir Isaac Newton determined his three laws of motions which led to new understanding in rocket science.

Rocket was still considered as weapons at this state and the accuracy was one of the major problems. The understanding and knowledge of aerodynamics was not yet established at this point. And therefore these rockets could not be used to carry a preservable payload, because of their accuracy and stability. 


From the 20th century and forward the use of rockets rapidly progressed and within the last 60 years rocket have been used for travelling and exploring our atmosphere and space. The development of this has been beneficial for humans, with the sending of satellites into orbit around earth and the scientific exploration of space. Today satellites and space travelling has a wide spread use for communication, scientific research and monitoring, which benefits most people on daily basis. This was not possible without the development of stable rockets that is able to carry a payload\cite{web:HistoryRocket}.


The early days problems of rockets was the short travelling distance. Therefore the focus was on having better thrust to gain further distances. The design of liquid fuel motors lead to that distance no longer was a problem. The main concern was the amount of fuel needed and the weight of the payload. The heavier the rocket is the more fuel it needs, and adding more fuel makes the rocket heavier. Incorrect distribution of the fuel could therefore make the rocket unstable, because of the change in the rockets center of gravity.


Stability and inaccuracy was also one of the major concerns in 19th century rockets. The rockets was designed with the thruster placed in the top/front, similar to how fireworks are designed today. This causes instability if the thruster is not powerful enough or the weight distribution is incorrect. As well does a disturbance as wind etc. effect the stability. But as we know from modern full scale rockets the thruster is placed at the bottom, which gives a higher risk of instability problems. When the thrust is applied from the bottom the rocket needs to be stable in the top. If the rocket becomes unbalanced at tip, the more likely it is to deviate from its path. 



\bigbreak
Incorrect weight distribution, faulty design, and aerodynamics is factors that all could lead to instability in rockets. These stability problems has all been faced throughout the history of rocket development, and will therefore be further examined to determine if a electronic control system can be implemented to improve the stability issues of rockets.



