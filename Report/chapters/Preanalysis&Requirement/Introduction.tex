\chapter{Introduction}\todo{Working on it - Mathias}
This project deals with the design of a stabilization system based on modelling of a rocket.
%within the atmosphere of Earth. 

Rockets is a technology used for hundreds of years. In the 13th century the Chinese and Mongols used fire based rocket, which had chemical mixed gunpowder as propulsion. Trough the 14th and 15th century the gunpowder improved and the rockets range increased, but was considered as weapons.  

The understanding of modern rockets was based on experiments with steam propulsion as with for example the aeolipile in the 17th century. Aeolipile is a rocket-like sphere, that rotates trough directing pressurised steam trough opposite pointing L-shaped nozzles. 

This meant that the approach to rockets became more scientific. In the later 17th century Sir Isaac Newton determined his three laws of motions which led to new understanding in rocket science.

Rocket was still considered as weapons at this state and the accuracy was one of the major problems. The understanding and knowledge of aerodynamics was not yet established at this point. And therefore these rockets could not be used to carry a preservable payload. 


From the 20th century and forward the use of rockets rapidly progressed and within the last 60 years rocket have been used for travelling and exploring our atmosphere and space. The development of this has been beneficial for humans, with the sending of satellites into orbit around earth and the scientific exploration of space. Today satellites has a wide spread use for communication, scientific research and monitoring and benefits most people on daily basis. This was not possible without the development of stable rockets that is able to carry a payload.   

Stability and inaccuracy was one of the major concerns in 19th century rockets. The rockets was designed with the thruster placed in the top/front, similar to how fireworks are designed today. This causes instability if the thruster is not powerful enough or the weight distribution is incorrect. As well does a disturbance as wind etc. effect the stability. But as we know from modern full scale rockets the thruster is placed at the bottom, which gives a greater instability problem. When the thrust is applied from the bottom the rocket needs to be stable in the top. If the rocket becomes unbalanced at tip, the more likely it is to deviate from its path.  



On the 15th of february 2017 an Indian rocket successfully launched with a payload of 104 satellites. The rocket beat the previous payload record for a Russian rocket carrying 37 satellites.  






Something about spacex and its stability problems.

Indian rocket satellites 

Low Efficiency of chemical rockets with payload    
The heavier payload the more fuel consumed.

The more fuel the rocket carries the more it needs. 

Pollution 


\textbf{Start with an explanation of why space exploration and orbitting technology is an advantage. Transition into how to put stuff in orbit or leave earth's gravity (rockets). Talk about issues with having basically a stick with thrusters beneath. It can tip over and variances in air currents have a large influence on the direction of the rocket. End with saying a control system that can make sure it's going straight is an advantage. Now we have a reason for making the control system we want.}


