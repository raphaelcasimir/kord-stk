\chapter{Preliminary Analysis of a Rocket}
The following chapter describes the functionality,  and structure behind a rocket. The goal is to determine which factors that leads to instability in flight and launch of a rocket. 



The full scale rocket model that will be described consist of,
\begin{itemize}[noitemsep]
\item a payload system.
\item a guidance system.
\item a propulsion system. 
\item a structural system.
\end{itemize}    
The model is seen cf. figure \ref{fig:RocketStructure}.
\begin{figure}[htbp]
	\centering
 	\includegraphics[width=0.65\textwidth]{figures/RocketStructure.png} 
 	\caption{Structure of a full scale rocket\cite{web:RocketStructure}.}
 	\label{fig:RocketStructure}
\end{figure}
%https://spaceflightsystems.grc.nasa.gov/education/rocket/rockpart.html

\textbf{Payload System}\\
Most rockets has some form of payload system. The goal of the payload system is to carry different objects to its wanted destination. The payload can be everything from satellites and astronauts to fireworks depending on the purpose of the rocket. The payload should be considered when looking at the stability of the rocket, because incorrect weight distribution in the rocket, could lead to aerodynamic and structural instability. 


\textbf{Guidance System}\\
All rockets that has the goal to be directed or controlled includes a guidance system. The guidance system consist of a processors, sensors, radars, and  a form of wireless communication. Its purpose is to control the stability, direction and rotation of the rocket during launch and in flight. The guidance system is developed based on the understanding of forces acting rocket and its motion. The guidance system in moderne rockets often actuate on the propulsion and nozzle system to correct rotation and direction of the rocket.

\textbf{Propulsion System}\\
The propulsion system of a rocket is the part which thrust the rocket. Thrust is the the main force that makes the rocket launch and fly. All propulsion systems is based on Newton's third law. This means that a propulsion system should be able make a combustion which produces a downwards force high enough to launch the structural system of the rocket.


The propulsion system can either be a liquid rocket engine or a solid rocket engine. A liquid rocket engine is based on a combustion of fuel and a oxidizer which is mixed an burned. The resulting gas of the burn, is directed trough a nozzle which accelerates it.
A solid rocket engine has premixed oxidizer and fuel which becomes the propellant. This propellant is compressed into a cylinder with an hole in it that functions as a combustion chamber. Which means that after ignition the propellants surface functions as the combustion chamber. The gas is therefore also forced trough a nozzle that accelerates it. Which applies i force to the engine that gives a launch of the rocket. 


\textbf{Structural System}\\
Close to all full scale rockets consist of a structural system. The system consist of the cylindrical body/frame, a nose cone with the payload system and the fins that ensures a stable aerodynamic profile. Though most newer full scales rocket does not rely only on aerodynamics to ensure stability\cite{web:RocketStructure}.


The correct combination of the system modules and design should ensure a stable flight, but as described cf. section \ref{sec:Introduction} it is shown that most rockets would benefit from a control system to ensure directional stability. Further investigation of the stability of rocket is proceeded. 


\subsection{Stability of a Rocket}
\graphicspath{{figures/"Preanalysis&Requirement"/RocketStability/}}
A rocket often has a target, either if it is space travelling or just collecting orbit data on earth, but in order to succeed it has to be stable \cite{web:rocketnasa}. The flight direction of a rocket can changed due to disturbances as wind and thrust instabilities.   

Stability can be defined within many categories. In the case of a rocket the stability is considered as the directional and flight stability.  

\autoref{fig:unstableRocketTrajectory} describes a example of a chaotic path for a badly designed rocket. Where the center and gravity and pressure is incorrect placed on the unstable rocket, which make it deviate from its goal trajectory.  

\begin{figure} [htbp]
	\centering
	\includegraphics[width=0.7\linewidth]{UnstableRocketTrajectory}
	\caption{Example of the path for an unstable rocket \cite{web:rocketnasa}.}
	\label{fig:unstableRocketTrajectory}
\end{figure}

Center of gravity and pressure is related to the forces acting on the rocket. An analysis of these forces is proceeded. \autoref{fig:RocketForceSummary} describes the different situations a rocket encounters during its flight and describes the forces applied in these cases.
\begin{figure}[htbp]
	\centering
	\includegraphics[width=0.7\linewidth]{ForceSummaryRocket}
	\caption{Summary of forces applied to a rocket during its flight \cite{web:rocketnasa}.}
	\label{fig:RocketForceSummary}
\end{figure}
\\
In \autoref{fig:RocketForceSummary} two points can be seen on the rockets. One is CP the Center of Pressure where the lift and the drag force apply \cite{web:rocketnasa} to a rocket. The other one is CG the Center of Gravity, which is the point which the rocket rotate around \cite{web:rocketnasa}. Due to this characteristics the torque applied by the lift and the drag is dependent of the relative positioning between the CG and the CP. Since a rocket tilts due to a torque applied by external forces such as the wind, then lift and the drag has to be oriented so it counters these forces. In order to do so the CG has to be above the CP \cite{web:rocketnasa}.
This means depending of the position of the CP compared to the CG the rocket flight will be either stable or unstable. That is why in simpler rockets without any control system, its design is made so that CP is indeed above CG.

This preliminary analysis makes gives the possibility to further examine on how to control the stability of a rocket.
%All based on https://spaceflightsystems.grc.nasa.gov/education/rocket/rockpart.html

\subsection{Controlling a Rocket's Stability}
The design method for placing the CP and GP does not always provide a perfectly stable attitude control. Indeed to have such precision an active control system is necessary. Full scale rockets use the thrusting force to achieve alike control. In order to do so most of them operate with the gimbaled thrust method. It consists of steering the engine's nozzle to get the thrusting force in right incidence. An example is seen cf. figure \autoref{fig:RocketGimbal}. 
\begin{figure} [htbp]
	\centering
	\includegraphics[width=0.8\linewidth]{RocketGimbal}
	\caption{Example of gimballing a rocket nozzle \cite{web:rocketnasa}.}
	\label{fig:RocketGimbal}
\end{figure}

Where a torque is applied to create a rotation around the rocket's center of gravity. The thrust direction is relative to the position of the center of gravity.  This should compensate for direction deviations from the rocket's center line or trajectory, and keep the rocket stable. The described gimbal method will be the control method focused on in this project. The rocket will be designed with a control system depending on vectoring the thruster in relation to the attitude position of the rocket. 


\section{Systems with known instability problems}
\graphicspath{{figures/"Preanalysis&Requirement"/SimilarSystems/}}
In order to stabilize the rocket in flight, a controller must be designed to stabilize the vertical position during flight.The possibility of damaging the rocket or the controller needs to be countered by the efficiency of the control system. Experimenting such a system during flight is expensive and presents nonnegligible hazards.
Is there a system that has similar instability properties of rockets, and presents fewer restrictions to experiment?

An inverted pendulum objective is to stabilize a stick whose center of mass is above pivot point. The process is done using a horizontal force, resulting in a first degree of freedom rotation. This is similar to a rocket system. An example of an inverted pendulum would be humans. Adjustments are needed to maintain balance when standing, walking, or running. Flight are not required, therefore it is less restrictive to use and analyze an inverted pendulum.

\begin{figure}[htbp]
	\centering
	\includegraphics[width=0.6\linewidth]{Inverted-Pendulum-exemple}
	\caption{Summary of forces applied to an inverted pendulum. The process is equivalent to figure  \vref{fig:RocketForceSummary}.}
	\label{fig:InvertedPendulum}
\end{figure}


The Cubli is another system known for its instability. It is commonly used in Control Theory to analyze the behavior of a reaction of a wheel inverted system. The advantages of the usage a Cubli is its ability to reach its equilibrium position in microgravity environments, such as asteroids. This is not relevant to the project.

A recent technology facing instability is Segway. A Segway is an application of an Inverted Pendulum to a two wheeled self-balance vehicle. This vehicle requires three body directions instead of the linear motion of an inverted pendulum. The study of instability properties of a Segway system would mean the study of elements nonexistent or negligible in a rocket system.

In consideration of the different applications and similarities of the three systems, the inverted pendulum will be analyzed to develop a model describing the dynamics of the system, and to design a control system stabilizing the position in vertical position. The objective is to understand and solve the instability properties shared by rocket and inverted pendulum systems.

