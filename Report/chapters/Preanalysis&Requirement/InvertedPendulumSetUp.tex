\chapter{Inverted Pendulum Analysis}\label{chap:IPAnalysis}


\section{Inverted Pendulum Description}\label{sec:IPDesc}
For this project an inverted pendulum setup is given. It consists of:
\begin{itemize}
	\item A DC motor
	\item A gear system
	\item An arm
	\item A stick connected to the end of the arm.
\end{itemize}
A diagram of the setup when fully assembled is seen on \autoref{fig:InvertedPendulumSetUp}.

\begin{figure} [htbp]
	\centering
	\includegraphics[width=0.6\linewidth]{figures/"Preanalysis&Requirement"/invertedPendulumDiagram}
	\caption{Diagram of the mechanical system\cite{web:BalancingStick2008}.} \label{fig:InvertedPendulumSetUp}
\end{figure}

The following describes the hardware connected to the inverted pendulum setup illustrated cf. figure \ref{fig:InvertedPendulumSetUp}.
Each part will be described with its specifications and use in the setup. The input and output relation of each block is illustrated cf. figure \ref{fig:DCMotorRelation}, where the goal troughout the chapter will be determining the transfer functions that describe these relations.

\begin{figure} [htbp]
\hspace*{-3.5cm}  
	\centering
	\includegraphics[width=0.95\paperwidth]{figures/modeling/InputOutputSystem.png}
	\caption{Input/output relation of the system plant.} \label{fig:DCMotorRelation}
\end{figure}

\startexplain
	\explain{$U_m$ is the motor input voltage}{\si{V}}
	\explain{$\omega_m$ is the angular velocity of the motor}{\si{\radian\per\second}}
	\explain{$\omega_w$ is the angular velocity of the gear connected to the arm}{\si{\radian\per\second}}
	\explain{$\tau_{f_{gear}}$ is the total friction of the gear system}{\si{\newton\meter}}
	\explain{$\tau_a$ is the load of the arm}{\si{\newton\meter}}
	\explain{$\tau_s$ is the load of the stick}{\si{\newton\meter}}
	\explain{$\theta_a$ is the angle from the arm to the y-axis}{\si{\radian}}
	\explain{$\theta_s$ is the angle from the stick to the y-axis}{\si{\radian}}
\stopexplain

\subsubsection{DC Motor}
A Axem servo motor model F9M2 is attached to the gears, this is to create a angular velocity in the system. To ensure control precision, the integrated tachometer of the motor is used as feedback. This is done so that it is possible to see the velocity and direction of the motor and relate it to the position of the arm. Using the DC motor also means implementing a motor controller. Already implemented in the setup is a Maxon Escon 50/5 Servocontroller, which can be controlled trough PWM. The servocontroller is supplied trough a 230 V regulator that can supply the servocontroller with up to 56 V and 15 A. The regulator in itself will be considered a blackbox, because of user limitations when working with 230 V at the university. The DC motor and servocontroller will in system diagrams be evaluated as one unit, but further considered when implementing a system controller.        

%http://www.maxonmotor.com/maxon/view/product/control/4-Q-Servokontroller/409510
	

\subsubsection{Gear System}
Inbetween the DC motor and arm is the gear system. The goal of the gear system is to reduce the ratio between the rotation of the motor versus the arm. The gear system is series of the same size small and big gear connected with belts. The setup is seen cf. figure \ref{fig:InvertedPendulumSetUp}, where the number of gear is illustrated. The parameters of the gear system is listed cf. table \ref{GearSystemParameters}.   

\begin{table}[htbp]
\centering
\begin{tabular}{llll}
\hline
Piece & Parameter & Value & Unit \\ \hline
Gear$_{big}$ & Teeth & 40 & {[}1{]} \\
Gear$_{big}$ & Diameter & 0.12 & {[}m{]} \\
Gear$_{small}$ & Teeth & 12 & {[}1{]} \\
Gear$_{small}$ & Diameter & 0.04 & {[}m{]} \\
Belt & Length & 0.6 & {[}m{]}
\end{tabular}
\caption{Parameters of gear system.}
\label{GearSystemParameters}
\end{table}

\subsubsection{Stick and Arm}
The arm and stick is elements that always appears in the double inverted pendulum setup. The goal is for the arm to apply force on their common joint, which would affect the position of the stick. The physical parameters for the stick and arm is listed cf. table \ref{DimensionsStick}.

\begin{table}[htbp]
\centering
\begin{tabular}{llll}
\hline
Piece           & Parameter & Value & Unit \\ \hline
Stick$_{long}$  & Length    & 0.8   & [m]    \\
Stick$_{long}$  & Weight    & 0.344 & [kg]    \\ 
Stick$_{short}$ & Length    & 0.4   & [m]    \\
Stick$_{short}$ & Weight    & 0.170 & [kg]    \\ 
Arm             & Length    & 0.33  & [m]    \\
Arm             & Weight    & 288   & [kg]   \\ 
\end{tabular}
\caption{Physical parameters of the arm and sticks.}
\label{DimensionsStick}
\end{table}
\newpage
Some feedback is needed to be able to control the stick trough the arm and gears. Sensors used is implemented as a integrated part of the setup, and will therefore be considered usable and will the choice of the will not be further discussed.

A necessary feedback to know is the angle between the stick and arm. This is needed to balance the stick via the controller. The angle between the stick and arm is detected and sampled from a potentiometers rotational position. Equally the position of the arm is needed, this is to determine the amount of change needed to counteract the change in the stick. The position of each potentiometer can be seen cf. figure \ref{fig:InvertedPendulumSetUpPotmeter}. 

\begin{figure} [htbp]
	\centering
	\includegraphics[width=0.35\linewidth]{figures/"Preanalysis&Requirement"/invertedPendulumWithPotmeter.png}
	\caption{Diagram of the arm and stick with illustrated sensors.} \label{fig:InvertedPendulumSetUpPotmeter}
\end{figure}
\newpage
The block diagram for the standard feedback control system with the known system plant is seen cf. figure \ref{fig:FeedbackSystem}. 

\begin{figure}[htbp]
\hspace*{-2.5 cm} 
	\centering
	\includegraphics[width=0.95\paperwidth]{figures/modeling/MechanicalSystem.PNG}
	\caption{Setup feedback loop.} \label{fig:FeedbackSystem}
\end{figure}
\todo{Might be changed, was not sure how we would handle the feedback of the sensors.}
\\

The basics of each system in the Inverted Pendulum is now described with its specifications, and the dynamics of the system can therefore be modelled.


\newpage



\section{Modelling of the Arm and Stick}\label{sec:StickArm}

%%%%%%%%%%%%% Equation template %%%%%%%%%%%%%%%
%\begin{flalign}
%\hspace{30pt} & EQUATION1 &&& \text{[UNIT]} \notag \\
%& EQUATION2 &&& \text{[UNIT]} \label{eq:LABEL} 
%\end{flalign}
%\begin{description}
%  \item[\hspace{30pt}\textnormal{where:}]\hfill \\
%  \begin{tabular}{p{30pt}lp{250pt}l}
%& $x$ & TEXT & [UNIT]  \\
%& $y$ & VERY LONG TEXT THAT IS VERY LONG AND HAS A LOT OF WORDS IN IT YET THE FORMATTING STILL LOOKS NICE AND CLEAN AND EVERYTHING IS AWESOME & [UNIT]  \\
%& $z$ & TEXT & [UNIT]
%\end{tabular}
%\end{description}
%%%%%%%%%%%%%%%%%%%%%%%%%%%%%%%%%%%%%%%%%%%%%%%
\graphicspath{{figures/modeling/ArmStick/}}
The goal of this section is to have a mathematical model for the behaviour of the angle of the stick in relation to the angle of the arm. The inputs and outputs of this system can be seen by the block diagram in \autoref{fig:StickBlock}.
\begin{figure}[htbp]
\centering
\includegraphics[width=0.4\textwidth]{InputOutputStick.pdf}
\caption{Block diagram of the inputs and outputs of the stick section of the inverted pendulum setup.}
\label{fig:StickBlock}
\end{figure}

The angles, constants and forces used to describe the system are seen on \autoref{fig:ArmStick}.
%\todo[inline,author=Jacob]{I'd like to somehow describe the process first instead of just going along randomly and suddenly ending up with the model. "We want to achieve this so we do that" etc.}
\begin{figure}[htbp]
\centering
\includegraphics[width=0.8\textwidth]{StickAndForces}
\caption{Diagram of the angles and forces acting on the arm and the stick.}
\label{fig:ArmStick}
\end{figure}

\startexplain
	\explain{$F_x$ is the force in the x direction}{\si{\newton}}
	\explain{$F_y$ is the force in the y direction}{\si{\newton}}
	\explain{$F_{px}$ is the x direction's reaction force for the arm and stick connection}{\si{\newton}}
	\explain{$F_{py}$ is the y direction's reaction force for the arm and stick connection}{\si{\newton}}
	\explain{$x_s$ is the position of the center of mass of the stick in the x direction}{\si{\meter}}
	\explain{$y_s$ is the position of the center of mass of the stick in the y direction}{\si{\meter}}
	\explain{$l_a$ is the length of the arm}{\si{\meter}}
	\explain{$l_s$ is the length of the stick}{\si{\meter}}
	\explain{$\theta_a$ is the angle from the arm to the y-axis}{\si{\radian}}
	\explain{$\theta_s$ is the angle from the stick to the y-axis}{\si{\radian}}
\stopexplain
\linebreak
All forces, constants and variables that relates to the arm and stick are denoted by a subscripted $a$ and $s$ respectively. 

The behaviour of the stick can be fully described by three movements; two translatory and one rotary. It can move in the x and y direction and rotate around its center of gravity. To fully describe the system two geometric equations are also needed.

%To find the relation between the angles, the free body diagram of the joint of the arm and stick is made on \autoref{fig:freebodystick}.
%\begin{figure}[htbp]
%\centering
%\includegraphics[width=0.25\textwidth]{FreeBodyPendulum}
%\caption{Free body diagram of the joint that connects the arm and the stick.}
%\label{fig:freebodystick}
%\end{figure}
%\todo[inline, author=Jacob]{Make pretty graph}
%
%The moment of inertia for the joint is described by \autoref{eq:Jsthetas}.
%\begin{subequations}
%\begin{flalign}
%& J_s\ddot{\theta}_s=\tau_s-\tau_f  \label{eq:Jsthetas} \\
%& \tau_s =F_p\frac{l_s}{2} \\
%& \tau_f =b_{as}\dot{\theta}_{as} 
%\end{flalign}
%\end{subequations}
%\startexplain
%	\explain{$J_s$ is the moment of inertia for the stick}{\si{\kg\square\meter}}
%	\explain{$\ddot{\theta}_s$ is the angular acceleration of the stick}{\si{\radian\per\square\second}}
%	\explain{$\tau_s$ is the torque induced by the rotation of the stick}{\si{\newton\meter}}
%	\explain{$\tau_f$ is the torque of the friction acting on the stick}{\si{\newton\meter}}
%	\explain{$F_p$ is the force perpendicular to the stick at the center of mass}{\si{\newton}}
%	\explain{$b_{as}$ is the viscous friction coefficient between the arm and the stick}{\si{\newton\meter\second}}
%	\explain{$\dot{\theta}_{as}$ is the difference in angular velocity between the arm and the stick ($\dot{\theta}_s-\dot{\theta}_a$)}{\si{\radian\per\second}}
%\stopexplain
%
%The friction is calculated from the difference in angular velocity as the stick could be perfectly upright while the arm moves causing the joint to turn. The angle of the arm is not considered as producing a torque acting on the joint but as part of the force on the stick, $F_p$.

The two translatory forces acting on the stick in the x and y directions are found by \autoref{eq:FxFy} using Newton's 2nd law of motion.
\begin{subequations}  \label{eq:FxFy}
\begin{flalign}
	& \ddot{x}_sM_s=F_x  \label{eq:transx} \\
	& \ddot{y}_sM_s=F_y-gM_s  \\
	& F_y=\left(\ddot{y}_s+g\right)M_s \label{eq:transy}
\end{flalign}
\end{subequations}
\startexplain
	\explain{$g$ is the standard gravitational acceleration near the surface of the earth}{\si{\meter\per\square\second}}
	\explain{$M_s$ is the mass of the stick}{\si{\kilo\gram}}
\stopexplain
The rotational movement of the stick is described by \autoref{eq:rotaryforce}.
\begin{flalign}
 J_s\ddot{\theta}_s &=\frac{l_s}{2}\left(F_x\cos(\theta_s)+F_y\sin(\theta_s)\right)-b_{as}\dot{\theta}_{as} \label{eq:rotaryforce}
\end{flalign}

Finally the position of the center of mass of the stick in the x and y direction is found by \autoref{eq:xsys} using geometry.
\begin{subequations}\label{eq:xsys} 
\begin{flalign}
& x_s=l_a\sin (-\theta_a)+\frac{l_s}{2} \sin (-\theta_s) \\
& x_s=-l_a\sin (\theta_a)-\frac{l_s}{2} \sin (\theta_s) \label{eq:geox} \\
& y_s = l_a\cos (-\theta_a)+\frac{l_s}{2} \cos(-\theta_s) \\
& y_s = l_a\cos (\theta_a)+\frac{l_s}{2} \cos(\theta_s) \label{eq:geoy}
\end{flalign}
\end{subequations}

\autoref{eq:transx}, \autoref{eq:transy}, \autoref{eq:rotaryforce}, \autoref{eq:geox} and \autoref{eq:geoy} are all the equations neccessary to fully describe the behaviour of the system. The goal of the modeling is to determine a transfer function that relates the angle of the stick to the angle of the arm. There are six unknown variables in the five equations: $F_x$, $F_y$, $x_s$, $y_s$, $\theta_a$ and $\theta_s$. It should therefore be possible to end with one equation with the two unknowns $\theta_a$ and $\theta_s$ i.e. the transfer function.

The derivatives of $x_s$ and $y_s$ are found in \autoref{eq:diffxy}.
\begin{subequations}\label{eq:diffxy} 
\begin{flalign}
\hspace{30pt} & \dot{x}_s=-l_a\dot{\theta}_a\cos(\theta_a)-\frac{l_s}{2}\dot{\theta}_s\cos(\theta_s) & [\si{\meter\per\second}] \\
& \ddot{x}_s=-l_a\ddot{\theta}_a\cos(\theta_a)+l_a\dot{\theta}_a^2\sin(\theta_a)-\frac{l_s}{2}\ddot{\theta}_s\cos(\theta_s)+\frac{l_s}{2}\dot{\theta}_s^2\sin(\theta_s) & [\si{\meter\per\square\second}] \\
& \dot{y}_s=-l_a \dot{\theta}_a\sin(\theta_a)-\frac{l_s}{2}\dot{\theta}_s\sin(\theta_s) & [\si{\meter\per\second}] \\
& \ddot{y}_s=-l_a\ddot{\theta}_a\sin(\theta_a)-l_a\dot{\theta}_a^2\cos(\theta_a)-\frac{l_s}{2}\ddot{\theta}_s\sin(\theta_s)-\frac{l_s}{2}\dot{\theta}_s^2\cos(\theta_s) & [\si{\meter\per\square\second}]
\end{flalign}
\end{subequations}

The forces $F_x$ and $F_y$ have an equal and opposite force at the point where the arm and stick connect. These can be decomposed into perpendicular and parallel forces. The parallel forces are negligible when assuming the stick is perfectly solid and unable to stretch or compress. The perpendicular forces are found by using geometry and show up in \autoref{eq:rotaryforce} for the rotary force and are seen on \autoref{fig:ArmStick}.

%\begin{figure}[htbp]
%\centering
%\includegraphics[width=0.25\textwidth]{ForcePerp}
%\caption{Diagram of the forces, $F_x$ and $F_y$, decomposed into perpendicular forces.}
%\label{fig:ForcePerp}
%\end{figure}

%\begin{subequations}\label{eq:perpFxFy}
%\begin{flalign}
%& F_{px}=F_x\cos(\theta_s) \\
%& F_{py}=F_y\sin(\theta_s)  \\
%& F_p = F_{px}+F_{py} 
%\end{flalign}
%\end{subequations}

The derivatives of the two geometric equations are inserted into \autoref{eq:transx} and \autoref{eq:transy} which are then inserted into \autoref{eq:rotaryforce} in \autoref{eq:JsLong}.
\begin{subequations}
\begin{flalign}
 J_s\ddot{\theta}_s &=\frac{l_s}{2}\left(\ddot{x}_sM_s\cos(\theta_s)+\left(\ddot{y}_s+g\right)M_s\sin(\theta_s)\right)-b_{as}\dot{\theta}_{as}  \\
 J_s\ddot{\theta}_s = \frac{l_s}{2}M_s \Big( &-l_a\ddot{\theta}_a\left(\cos(\theta_a)\cos(\theta_s)+\sin(\theta_a)\sin(\theta_s)\right) \notag \\
& +l_a\dot{\theta}_a^2\left(\sin(\theta_a)\cos(\theta_s)-\cos(\theta_a)\sin(\theta_s)\right) \notag \\
& -\frac{l_s}{2}\ddot{\theta}_s\left(\cos(\theta_s)\cos(\theta_s)+\sin(\theta_s)\sin(\theta_s)\right) \notag \\
& +\frac{l_s}{2}\dot{\theta}_s^2\left(\sin(\theta_s)\cos(\theta_s)-\cos(\theta_s)\sin(\theta_s)\right)  \notag \\
& +g\sin(\theta_s) \Big)-b_{as}\dot{\theta}_{as} \label{eq:JsLong}
\end{flalign}
\end{subequations}

Using the trigonometric properties in \autoref{eq:trigprop}, \autoref{eq:JsLong} is reduced to \autoref{eq:JsShort}.
\begin{subequations} \label{eq:trigprop}
\begin{flalign}
& \cos(\theta_a)\cos(\theta_s)\pm \sin(\theta_a)\sin(\theta_s)=\cos(\theta_a \mp \theta_s)  \\
& \sin(\theta_a)\cos(\theta_s)\pm \cos(\theta_a)\sin(\theta_s) = \sin(\theta_a \pm \theta_s) \\ 
& \cos(\theta_s)^2+\sin(\theta_s)^2=1 
\end{flalign}
\end{subequations}
\begin{flalign}
J_s\ddot{\theta}_s = \frac{l_s}{2}M_s \Big( &-l_a\ddot{\theta}_a \cos(\theta_a-\theta_s)+l_a\dot{\theta}_a^2 \sin(\theta_a-\theta_s) \notag \\
&-\frac{l_s}{2}\ddot{\theta}_s +g\sin(\theta_s) \Big)-b_{as}\dot{\theta}_{as} \label{eq:JsShort}
\end{flalign}

This is the nonlinear mathematical model for the system. This will be linearized in order to perform a Laplace transformation. The linearization can be seen cf. \autoref{sec:LinearStick}.

The linearized model is \autoref{eq:LinStick}.
\begin{flalign}
& J_s\ddot{\theta}_s=\frac{l_s}{2}M_s\left(-l_a\ddot{\theta}_a-\frac{l_s}{2}\ddot{\theta}_s+g\theta_s\right)-b_{as}\dot{\theta}_{as} \label{eq:LinStick}
\end{flalign}

Inserting the moment of inertia for a rotating stick, $J_s=\frac{1}{12}M_sl_s^2$, the linearized model becomes \eqref{eq:JsFinal} \cite{web:MInertia}.
\begin{subequations}
\begin{flalign}
& \frac{1}{12}M_sl_s^2\ddot{\theta}_s=\frac{l_s}{2}M_s\left(-l_a\ddot{\theta}_a-\frac{l_s}{2}\ddot{\theta}_s+g\theta_s\right)-b_{as}\dot{\theta}_{as}   \\
& \frac{1}{12}M_sl_s^2\ddot{\theta}_s+\frac{1}{4}M_sl_s^2\ddot{\theta}_s=\frac{l_s}{2}M_s\left(-l_a\ddot{\theta}_a+g\theta_s\right)-b_{as}\dot{\theta}_{as}   \\
& \frac{1}{3}M_sl_s^2\ddot{\theta}_s=\frac{l_s}{2}M_s\left(-l_a\ddot{\theta}_a+g\theta_s\right)-b_{as}\dot{\theta}_{as}  \label{eq:TauSmLin} \\
& \ddot{\theta}_s=\frac{3}{2l_s}\left(-l_a\ddot{\theta}_a+g\theta_s\right)-\frac{3b_{as}\left(\dot{\theta}_s-\dot{\theta}_a\right)}{M_sl_s^2} \label{eq:JsFinal}
\end{flalign}
\end{subequations}

The linearized model is now Laplace transformed in \autoref{eq:tfArmStick} in order to find the transfer function.
\begin{subequations}
\begin{flalign}
& s^2\Theta_s=\frac{3}{2l_s}\left(-s^2l_a\Theta_a +g\Theta_s\right)-s\frac{3b_{as}}{M_sl_s^2}\Theta_s+s\frac{3b_{as}}{M_sl_s^2}\Theta_a  \\
& \Theta_s\left(s^2+\frac{3b_{as}}{M_sl_s^2}s-\frac{3g}{2l_s}\right)=\Theta_a\left(-\frac{3l_a}{2l_s}s^2+\frac{3b_{as}}{M_sl_s^2}s\right)  \\
& \frac{\Theta_s}{\Theta_a}=\frac{-\frac{3l_a}{2l_s}s^2+\frac{3b_{as}}{M_sl_s^2}s}{s^2+\frac{3b_{as}}{M_sl_s^2}s-\frac{3g}{2l_s}} \label{eq:tfArmStick}
\end{flalign}
\end{subequations}

The system has a zero in 0 (two if the friction is considered negligible) which gives a 0 dB DC gain. This makes sense as the stick should not move if the angle of the arm is constant. The poles show that the natural frequency of the system depends only on the gravity and length of the stick. This is similar to \autoref{eq:pendulum} for the frequency of a simple pendulum \cite{web:Pendulum}.
\begin{flalign}\label{eq:pendulum}
\omega_n=\sqrt{\frac{g}{L}}
\end{flalign}

A linearized model in the Laplace domain for the arm and the stick has been derived and the model for the motor and gears is now derived. The load torque generated by the arm and stick is significantly smaller than the torque of the gear system, at the DC motor, and is therefore assumed negligible. Equally the friction between the arm and the stick is assumed to be zero as the joint consists of a ball bearing. 
