\chapter{Requirements and Specifications}
The following chapter describes requirements and specifications that is determined for controlling both the rocket and the inverted pendulum. The requirements is set to obtain a system which fulfils the problem statement \todo{Ref to section - Mathias and might be changed in formulation}.

As described earlier the goal of both systems is to react to deviations from its stable position. The goal of the rocket is the launch and fly stable and straight versus the inverted pendulum where the goal is to balance the stick in vertical upwards position.  

Both systems will be described with physical and control requirements in separate sections. The physical requirements will be set based on the limits of the models. Where the requirements for the controllers will be set based on the approximations of best possible control parameters for each model.  
The controller requirements will be approximated based on speed, precision and stability, which means that following control parameters would be set for:

\begin{itemize}[noitemsep]
\item Settling Time
\item Overshoot
\item Rise Time
\item Steady State Error
\end{itemize}

Each parameter would affect each other, and therefore a discussion of the best combination of them will be considered.

\subsection{Requirements for the Inverted Pendulum}
The requirements is based on figure \ref{fig:InvertedPendulumSetUp} and the modeling behind where the angle relations and positions is described between the gears, arm and stick. 

\setlength{\parindent}{0pt}
\newcommand{\forceindent}{\leavevmode{\parindent=1em\indent}}

\subsubsection*{1. The control position of the arm must not exceed $\pm$ $\dfrac{\pi}{4}$ radians from the initial position.}
\forceindent A choice made to limit the arms movements when balancing the stick.


\subsubsection*{2. The control position of the stick must not exceed $\pm$ $\dfrac{\pi}{18}$ radians from the initial position.}
\forceindent An assumption made to simplify and limit the sticks movement, which would relate it to the control of a rocket.
\todo{Changed based on test results.}
\subsubsection*{3. Deviation from the upright position of the stick must not exceed $\pm$  $\dfrac{\pi}{90}$ radians when considered balanced.}
\forceindent A choice made to ensure stability and avoid oscillation around the equilibrium position.
\todo{Might be changed based on resolution and precision of potentiometer}

\subsubsection*{4. The system should be able to regain balance if an impulse of $\dfrac{\pi}{18}$ radians is applied, in form of a push on the stick.} 
\forceindent If an impulse from the external disturbance is acting on the stick then the control should be able to counteract and force the stick back into equilibrium position. 


Control requirements:



\subsection{Requirements for the Rocket}
Some limits has to be set based on the capability of building and using a rocket.  
The physical dimensions of the rocket can not exceed:
\begin{table}[htbp]
\centering
\begin{tabular}{lll}
\hline
Parameter    & Value & Unit  \\ \hline
Length       & 0.25  & [m]     \\
Width        & 0.25  & [m]     \\
Height       & 0.5   & [m]     \\
Total volume & 0.03  & [m$^3$] \\
Weight       & 0.3   & [kg]   
\end{tabular}
\caption{Maximum size and weight of the rocket.}
\label{RocketDimensions}
\end{table}


\paragraph{5. Deviation from the rockets initial trajectory must not exceed $\pm$ $\dfrac{\pi}{18}$ radians.} \todo{Might be changed based on resolution and precision of gyroscope}

\forceindent A choice made to limit the movement of the rocket during flight. If the limit is exceeded the rocket controller will not guarantee stability, but might stabilize the rocket none the less.      

\paragraph{6. The system should be able to regain its stability and direction if an external disturbance impulse of $\dfrac{\pi}{36}$ radians is applied on the rocket.}\todo{Maybe a time span?}  

\forceindent  A choice made to ensure flight stability even if the rocket is influenced by a external disturbance.

\section{Acceptance Test Specification}	
TBD