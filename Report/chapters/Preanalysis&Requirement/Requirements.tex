\chapter{Requirements and Specifications}
The following chapter describes requirements and specifications determined for controlling both the rocket and the inverted pendulum. The requirements are set to obtain a system which fulfils the problem statement \todo{Ref to section - Mathias and might be changed in formulation}.

As described previously the goal of both systems is to react to deviations from its stable position. The goal of the rocket is the launch and a stable flight, versus the inverted pendulum where the goal is to balance the stick in vertical upwards position.  

Both systems will be described with physical and control requirements in separate sections. The physical requirements will be set based on the limits of the models. 
%The requirements for the controllers will be set based on the approximations of best possible control parameters for each model.  
The controller requirements will be based on speed, precision and stability, which means that control parameters would be set for:

\begin{itemize}[noitemsep]
\item Settling Time
\item Overshoot
\item Rise Time
\item Steady State Error
\end{itemize}

Each parameter would affect each other, and therefore a discussion of the best combination will be considered.
\newpage
\subsection{Requirements for the Inverted Pendulum}
The requirements are based on figure \ref{fig:InvertedPendulumSetUp}, and the modeling behind where the angle relations and positions is described between the gears, arm and stick. It is yet decided that the main requirement set is to balance the stick. But requirements have to be set considering when the stick is balanced and when not. \todo{The figure can be added, to simplify the explanation.}

\setlength{\parindent}{0pt}
\newcommand{\forceindent}{\leavevmode{\parindent=1em\indent}}

\subsubsection*{1. The control position of the arm must not exceed $\pm$ $\dfrac{\pi}{4}$ radians from the 0 radians vertical position.} 
\forceindent A choice made to limit the arms movements around vertical upwards position which will act as the initial position. The arm should not exceeded the limit because the control of the stick will become more vertical than horizontal. It is therefore considered that the arm is not able to control the stick in a stable manor.  


\subsubsection*{2. The angle of the stick must not exceed $\pm$ $\dfrac{\pi}{18}$ radians from vertical position.}
\forceindent An assumption made to simplify and limit the sticks movement, which would relate it to the control of a rocket. Assuming that the arm can not move to more than $\dfrac{\pi}{4}$, gives that the overshoot of the arm can be at maximum:


\begin{equation}
\dfrac{\pi}{4}\ -\ \dfrac{\pi}{18}\ = \dfrac{7\pi}{36}
\end{equation}
\begin{equation}
\frac{\dfrac{7\pi}{36}}{\dfrac{\pi}{18}}\ =\ 350\ \% 
\end{equation}

This only apply when considering the worst case position of the stick when the arm is still in vertical position. It means that arm should be able to catch the stick without overshooting more than 350\%. The overshoot is limit, because 350\% is considered unstable when the goal is to balance the stick. The maximum overshoot is limited to a tenth of the sticks angular position $\dfrac{\pi}{180}$. And may therefore not exceed:

\begin{equation}
\frac{\dfrac{\pi}{180}}{\dfrac{\pi}{18}}\ =\ 10\ \% 
\end{equation}


\subsubsection*{3. Deviation from the upright position of the stick must not exceed $\pm$ $\dfrac{\pi}{180}$ radians when considered balanced.}
\forceindent A choice made to ensure stability and avoid oscillation around the equilibrium position.
This means that if the angle is exceeded the controller needs to react. The requirement is mainly set to ensure that the stick balancing precision is within the sampling precision of the sensors. The controller will be designed based on a steady state error of zero, which means that errors from balanced position are from disturbance and not from the controller's steady state error.

\subsubsection*{4. The system should be able to regain balance if an impulse of $\dfrac{\pi}{18}$ radians is applied, in form of a push on the stick.} 
\forceindent If an impulse from the external disturbance is acting on the stick then the control should be able to counteract the stick back into equilibrium position.   



\subsection{Requirements for the Rocket}
Some limits has to be set based on the capability of building and using a rocket. The controller requirements will be the same as with the inverted pendulum.  
The physical dimensions of the rocket can not exceed:
\begin{table}[htbp]
\centering\begin{flushleft}
	\begin{flushleft}
		\begin{flushleft}
			\begin{flushleft}
				\begin{flushleft}
					\begin{flushleft}
						\begin{flushleft}
							\begin{flushleft}
								
							\end{flushleft}
						\end{flushleft}
					\end{flushleft}
				\end{flushleft}
			\end{flushleft}
		\end{flushleft}
	\end{flushleft}
\end{flushleft}
\begin{tabular}{lll}
\hline
Parameter    & Value & Unit  \\ \hline
Length       & 0.25  & [m]     \\
Width        & 0.25  & [m]     \\
Height       & 0.5   & [m]     \\
Total volume & 0.03  & [m$^3$] \\
Weight       & 0.3   & [kg]   
\end{tabular}
\caption{Maximum size and weight of the rocket.}
\label{RocketDimensions}
\end{table}


\subsubsection*{5. Deviation from the rockets initial trajectory must not exceed $\pm$ $\dfrac{\pi}{18}$ radians.} \todo{Might be changed based on resolution and precision of gyroscope}

\forceindent A choice made to limit the movement of the rocket during flight. If the limit is exceeded the rocket controller will not guarantee stability, but might stabilize the rocket none the less. 

\subsubsection*{6. The system should be able to regain its stability and direction if an external disturbance impulse of $\dfrac{\pi}{36}$ radians is applied on the rocket.} 

\forceindent  A choice made to ensure flight stability even if the rocket is influenced by a external disturbance.

\section{Acceptance Test Specification}	
The following section describes how the requirements can be tested. This verification is made to ensure that the system fulfils the set requirements.

\subsubsection*{Acceptance Test 1.}

\forceindent Verification of requirement 1 will not be proceeded, but simply implemented in the software based on feedback from the arm potentiometer. The software will, if the angle exceeds $\pm$ $\dfrac{\pi}{4}$ radians, set the arm to initial vertical position and no longer react on feedback from the stick, until the stick has been balanced around $\pm$ $\dfrac{\pi}{18}$ radians in at least 5 seconds. The controller will then again react to changes on the stick.    

\subsubsection*{Acceptance Test 2.}

\forceindent Requirement 2 and 3 is tested based on placing the stick and arm in vertical balanced position. The controller is engaged and the system is ran for ten minutes. The feedback data of the sensors will be collected to determine if the sticks angle have exceeded the requirement. As well is the sampled data from the motor used to compare with the position changes of the stick and see if the controller consider the stick balanced when the angle of the stick is within $\pm$ $\dfrac{\pi}{180}$ of 0 radians. 

\subsubsection*{Acceptance Test 3.}
\forceindent Requirement 4 is tested from holding the stick in $\pm\ \dfrac{\pi}{18}$ radians when the arm is at 0 radians vertical position. The control system is activated and the stick is released. The test is repeated 5 times and the sensor data is sampled and saved each time.   

\todo{Mathias - Add rocket test if it will be testable}