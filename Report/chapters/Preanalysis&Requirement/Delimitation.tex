\chapter{Problem Statement \& Delimitation}
This chapter will include the problem statement and delimitations in the project.

\section{Problem Statement}
It is chosen to separate the design of controller into two separate persons, considering that the model of the inverted pendulum and the rocket do not share an overall similarity. A problem statement for each part is introduced.

\bigbreak
\textit{How can a control system be designed for implementation in the inverted pendulum system? And would this system be capable of balancing the stick?}

\bigbreak
\textit{How can a control system be designed for implementation on the rocket? And is it possible to stabilize the rockets trough thrust vectoring with this system?}

\section{Delimitation}
In the project some limits have to be set for the use of models and materials. The objective of the control system is to find the similarities of using a electronic stabilization system for rockets and inverted pendulums. The system is developed as a proof-of-concept solution. The models in the project can only be approximations of reality, and it is therefore observable that the transfer to a larger scale project will not be linear.        

\textbf{Physical delimitation:}
The inverted pendulum setup used in the project is pre-fabricated and made available by the university. The choice of motor, gears, and other components will therefore not be further considered. The setup is a inverted pendulum, and contains therefore of both an arm and a stick. The inverted pendulum setup will be from here on be called "inverted pendulum". The setup is shown cf. figure \ref{fig:InvertedPendulum1}.
\begin{figure}[htbp]
	\centering
	\includegraphics[width=0.7\linewidth]{figures/"Preanalysis&Requirement"/invertedPendulumDiagram}
	\caption{Diagram of the inverted pendulum setup\citep{web:BalancingStick2008}.} \label{fig:InvertedPendulum1}
\end{figure}


Construction and modeling of the rocket will be limited to a minor-scale rocket. The rocket will be designed around a solid propellant thruster, and will therefore not contain any liquid fuel. This is to limit the constraints from laws and keep the cost low. Furthermore weight and dimension limits are set before designing the rocket.
 
 
\textbf{Control delimitation:}
A choice is made that the starting point for controlling the stick is as illustrated cf. figure \ref{fig:InvertedPendulum1} in upwards vertical position. Therefore the controller will not be able to balance the stick, if its vertical balance limits is surpassed. The delimitation is made to simplify the controlling, and make it as similar as possible to controlling a rocket. 
 
 
\textbf{Test delimitation:}
Launching and flight of a rocket for testing a controller is a high cost procedure because of the chance of damaging the rocket and the cost of thrusters per launch. The testing of the rockets stabilization will be a ground based test setup, where the controller can be tested without the risk of damaging the rocket. If, and only if the control system proves stable and safe, a launch and flight of the rocket will be conducted under circumstances fulfilling given laws.   
\bigbreak


The delimitation now gives the possibility to set requirements and specifications for the established models for both systems.