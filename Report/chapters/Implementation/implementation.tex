\chapter{Implementation}
This chapter includes the implementation of controllers and components. The chapter will be seperated into two section, a section for the inverted pendulum and another section for the rocket. An Arduino will be used as the platform to implement the hardware and software on.    


\subsection{MCU}\label{sec:MCU}
TBD

\todo{Sampling speed for both system, and argument that the arduino is fine with its 16 MHz clock speed.}

\section{Inverted Pendulum Implementation}
This section describes the implementation of the controller design cf. section \ref{sec:IPController}

\subsection{Implementing Sensors}

\subsubsection*{Potentiometer}

\subsubsection*{Tachometer}
The tachometer and its precision has been tested in appendix \ref{appendix:RPMTest}. The test concluded that the external optic A2108 tachometers precision is within the limit of what could implemented. The tachometer outputs a voltage which is linear with the number of revolutions per minute. It can be used in two modes, one where 1 V = 1000 rpm and one where 1 V = 10.000 rpm. This voltage can be used as an analogue input to the Arduino and then converted trough software. An example is seen in the following software sample: 

\begin{lstlisting}
void loop() {
  // read the input on analog pin 0:
  int TachoMeter = analogRead(A0);
  int VoltageTachoMeter = map(TachoMeter, 1, 1023, 0, 5) //Map the analog value back to a voltage so that the conversion is easier.
  int RadS = (VoltageTachoMeter * 1000) * 0.104719755 //Convert from RPM to rad/s
  Serial.println(RadS);
  delay(1);        // delay in between reads for stability
}
\end{lstlisting}  

Which gives a transfer function of:
\begin{equation}
TBD
\end{equation}

\subsection{Implementing Motor Controller}
TBD


\section{Rocket Implementation}
TBD

\subsection{Implementing Inertial Measurement Unit}